\documentclass[11pt,a4paper]{article}

\usepackage{xcolor}
\definecolor{bg}{rgb}{0.96,0.96,0.96}
\usepackage{minted}
\setminted[]{fontsize=\small,bgcolor=bg}

\usepackage{hyperref}

\usepackage[utf8]{inputenc}
\usepackage{indentfirst}

\usepackage[round,semicolon]{natbib}
\bibliographystyle{plainnat}

\title{\textbf{JupyterHub@JSC evaluation}}

\author{
  Martin Claus \\ % alphabetical order
	Katharina Höflich \\
	Willi Rath}

\begin{document}

\maketitle
\tableofcontents

%==============================================================================

% STYLE GUIDE
%
% - One sentence per line. This avoids merge conflicts and creates easier to read git diffs.
% - TBC.

\section{Introduction}
\label{s-introductoin}

%==============================================================================

\section{Documentation from user perspective}
\label{s-doc-from-user-pov}

\begin{itemize}
	\item Are the docs up to date with the system that is offered?
	\item Are the docs up to date with the Jupyter docs?
	\item Is there a way of contributing to the docs?
	\item Is it easy for new Jupyter users to get started just based on what's provided by the centre?
\end{itemize}

%==============================================================================
\citet{Lorem2020a} %local PDF build workaround
\section{Working environments}
\label{sect:working-envs}

In this section a description of the software environments accessible via the JupyterHub system is given.
PUT DESCRIPTION OF FURTHER SECTION CONTENTS HERE.

\subsection{JupyterLab system description}
\label{sect:system-description}

JupyterLab sessions on JUWELS nodes are instantiated based on activating the following hierarchy of \href{https://lmod.readthedocs.io/en/latest/index.html}{Lmod software environment modules}

\begin{minted}[breaklines,breakanywhere]{bash}
$ module purge --force
$ module use $OTHERSTAGES
$ module load Stages/Devel-2019a GCC/8.3.0
$ module load Jupyter/2019a-Python-3.6.8
$ which jupyter
/gpfs/software/juwels/stages/Devel-2019a/software/Jupyter/2019a-gcccoremkl-8.3.0-2019.3.199-Python-3.6.8/bin/jupyter
\end{minted}

Executing the following command on JUWELS login nodes

\begin{minted}[breaklines,breakanywhere]{bash}
$ jupyter lab --ip=$HOSTNAME --no-browser
\end{minted}

starts up the same JupyterLab instance that is otherwise spawned by the JupyterHub system.
The following software environment modules\footnote{the command must be executed before the Jupyter module is loaded}

\begin{minted}[breaklines,breakanywhere]{bash}
$ module --redirect show Jupyter/2019a-Python-3.6.8 | grep "load("
load("imkl/.2019.3.199")
load("Python/3.6.8")
load("SciPy-Stack/2019a-Python-3.6.8")
load("libyaml/.0.2.2")
load("cling/.0.6dev")
load("pandoc/2.7.2")
load("texlive/2018")
load("Julia/1.1.0")
load("ITK/5.0.1-Python-3.6.8")
load("HDF5/1.10.5-serial")
load("netcdf4-python/1.5.3-serial-Python-3.6.8")
load("FFmpeg/.4.1.3")
\end{minted}

are further activated by the \verb|Jupyter/2019a-Python-3.6.8| and provide the default software environment that is used by the following Jupyter kernels

\begin{minted}[breaklines,breakanywhere]{bash}
$ jupyter kernelspec list
Available kernels:
  bash          /gpfs/software/juwels/stages/Devel-2019a/software/Jupyter/2019a-gcccoremkl-8.3.0-2019.3.199-Python-3.6.8/share/jupyter/kernels/bash
  cling-cpp11   /gpfs/software/juwels/stages/Devel-2019a/software/Jupyter/2019a-gcccoremkl-8.3.0-2019.3.199-Python-3.6.8/share/jupyter/kernels/cling-cpp11
  cling-cpp14   /gpfs/software/juwels/stages/Devel-2019a/software/Jupyter/2019a-gcccoremkl-8.3.0-2019.3.199-Python-3.6.8/share/jupyter/kernels/cling-cpp14
  cling-cpp17   /gpfs/software/juwels/stages/Devel-2019a/software/Jupyter/2019a-gcccoremkl-8.3.0-2019.3.199-Python-3.6.8/share/jupyter/kernels/cling-cpp17
  javascript    /gpfs/software/juwels/stages/Devel-2019a/software/Jupyter/2019a-gcccoremkl-8.3.0-2019.3.199-Python-3.6.8/share/jupyter/kernels/javascript
  julia-1.1     /gpfs/software/juwels/stages/Devel-2019a/software/Jupyter/2019a-gcccoremkl-8.3.0-2019.3.199-Python-3.6.8/share/jupyter/kernels/julia-1.1
  python3       /gpfs/software/juwels/stages/Devel-2019a/software/Jupyter/2019a-gcccoremkl-8.3.0-2019.3.199-Python-3.6.8/share/jupyter/kernels/python3
\end{minted}

and that can be chosen in the JupyterLab launcher tab.

\subsubsection{Lmod software Jupyter extension}

The default JupyterLab instance software environment can directly be modified with the \href{https://github.com/cmd-ntrf/jupyter-lmod}{Jupyter Lmod extension} that is accessible via the JupyterLab sidebar.
The currently activated default software environment modules are listed, as well as every module available in the \verb|Devel-2019a| stage and \verb|GCC/8.3.0| compiler module hierarchy, which are those modules also listed by the \verb|module avail| command in a shell.

Using the Lmod softwares extensions approach, the default Python environment can, for example, be extended by loading the \verb|Keras/2.2.4-Python-3.6.8| software module (also available in a GPU variant) that automatically sets up the necessary \verb|TensorFlow/1.13.1-Python-3.6.8| backend. After restarting the Python kernel the following environment changes

\begin{minted}{bash}
$ pip list > default.txt
$ module load Keras/2.2.4-Python-3.6.8
$ pip list > changed.txt
$ diff default.txt changed.txt
3c3
< absl-py                            0.8.1
---
> absl-py                            0.7.1
14a15
> astor                              0.7.1
72a74
> gast                               0.2.2
76a79
> grpcio                             1.20.1
144a148,150
> Keras                              2.2.4
> Keras-Applications                 1.0.7
> Keras-Preprocessing                1.0.9
153a160
> Markdown                           3.1
277a285,288
> tensorboard                        1.13.1
> tensorflow                         1.13.1
> tensorflow-estimator               1.13.0
> termcolor                          1.1.0
\end{minted}

are applied and available inside Python kernel notebooks.

While for the previous example applied software environment changes are successful, conflicting examples can also be identified.
By loading e.g. the \verb|numba/0.43.1-Python-3.6.8| module the following changes

\begin{minted}{bash}
$ module load numba/0.43.1-Python-3.6.8
The following have been reloaded with a version change:
  1) LLVM/8.0.0 => LLVM/7.0.1-dev
$ pip list > changed.txt
$ diff default.txt changed.txt
149c149
< llvmlite                           0.30.0
---
> llvmlite                           0.28.0
$ pip show numba | grep -i version
Version: 0.46.0
\end{minted}

are applied, that is activation of a previous version of the LLVM compiler infrastructure and the llvmlite Python package.
However, the numba package version that is accessed via the Jupyter kernel does not change.
Evaluation of the Python package environment

\begin{minted}[breaklines,breakanywhere]{bash}
$ module --redirect show numba/0.43.1-Python-3.6.8 | grep PYTHONPATH
prepend_path("PYTHONPATH","/gpfs/software/juwels/stages/Devel-2019a/software/numba/0.43.1-gcccoremkl-8.3.0-2019.3.199-Python-3.6.8/lib/python3.6/site-packages")
$ pip show numba | grep -i location
Location: /gpfs/software/juwels/stages/Devel-2019a/software/Jupyter/2019a-gcccoremkl-8.3.0-2019.3.199-Python-3.6.8/lib/python3.6/site-packages
$ printenv | grep -i pythonpath
PYTHONPATH=/gpfs/software/juwels/stages/Devel-2019a/software/numba/0.43.1-gcccoremkl-8.3.0-2019.3.199-Python-3.6.8/lib/python3.6/site-packages:/gpfs/software/juwels/stages/Devel-2019a/software/Jupyter/2019a-gcccoremkl-8.3.0-2019.3.199-Python-3.6.8/lib/python3.6/site-packages:/gpfs/software/juwels/stages/Devel-2019a/software/netcdf4-python/1.5.3-gcccoremkl-8.3.0-2019.3.199-serial-Python-3.6.8/lib/python3.6/site-packages:/gpfs/software/juwels/stages/Devel-2019a/software/ITK/5.0.1-gcccoremkl-8.3.0-2019.3.199-Python-3.6.8/lib/python3.6/site-packages:/gpfs/software/juwels/stages/Devel-2019a/software/VTK/8.2.0-gcccoremkl-8.3.0-2019.3.199-Python-3.6.8/lib64/python3.6/site-packages:/gpfs/software/juwels/stages/Devel-2019a/software/SciPy-Stack/2019a-gcccoremkl-8.3.0-2019.3.199-Python-3.6.8/lib/python3.6/site-packages:/gpfs/software/juwels/stages/Devel-2019a/software/Python/3.6.8-GCCcore-8.3.0/easybuild/python:/gpfs/software/juwels/stages/Devel-2019a/software/Python/3.6.8-GCCcore-8.3.0/lib/python3.6/site-packages
$ ls /gpfs/software/juwels/stages/Devel-2019a/software/Jupyter/2019a-gcccoremkl-8.3.0-2019.3.199-Python-3.6.8/lib/python3.6/site-packages -lrtha | grep -i numba
drwxr-sr-x  24 goebbert1 swmanage  32K 25. Jan 09:24 numba
drwxr-sr-x   2 goebbert1 swmanage 4,0K 25. Jan 09:24 numba-0.46.0.dist-info
$ ls /gpfs/software/juwels/stages/Devel-2019a/software/numba/0.43.1-gcccoremkl-8.3.0-2019.3.199-Python-3.6.8/lib/python3.6/site-packages -lrtha | grep -i numba
drwxrwsr-x 4 swmanage swmanage 4,0K  9. Mai 2019  numba-0.43.1-py3.6-linux-x86_64.egg
\end{minted}

reveals that numba from the \verb|Jupyter/2019a-Python-3.6.8| environment is used, even though the \verb|PYTHONPATH| is correctly set.
Generally, such unexpected software module environment behaviour is not desired and could lead to conflicting software with broken Jupyter kernel functionality.

\subsubsection{Jupyter kernel functionality}

Indeed, by using the Lmod software modules extension users can completely destroy their default Jupyter kernel functionality, and even the functionality of the whole JupyterLab instance.

The Lmod softwares Jupyter extension allows, for example, replacing the default \verb|Julia/1.1.0| with a newer \verb|Julia/1.3.1| software environment.
This, however, breaks Julia kernel functionality.
Checking the Jupyter kernel configuration file

\begin{minted}[breaklines,breakanywhere]{bash}
$ cd /gpfs/software/juwels/stages/Devel-2019a/software/Jupyter/2019a-gcccoremkl-8.3.0-2019.3.199-Python-3.6.8/share/jupyter/kernels/julia-1.1
$ cat kernel.json
{
  "display_name": "Julia 1.1.0",
  "argv": [
    "/gpfs/software/juwels/stages/Devel-2019a/software/Julia/1.1.0-gcccoremkl-8.3.0-2019.3.199/bin/julia",
    "-i",
    "--startup-file=yes",
    "--color=yes",
    "--project=@.",
    "/gpfs/software/juwels/stages/Devel-2019a/software/Jupyter/2019a-gcccoremkl-8.3.0-2019.3.199-Python-3.6.8/share/julia/site/packages/IJulia/F1GUo/src/kernel.jl",
    "{connection_file}"
  ],
  "language": "julia",
  "env": {},
  "interrupt_mode": "signal"
}
\end{minted}

reveals that the Julia path is hard coded to the 1.1.0 version.
Replacing the software environment paths with those of Julia 1.3.1, but calling Julia 1.1.0 apparently leads to conflicts that then cause a dead kernel.
The JupyterLab system does not provide error messages on this.

Currently, the Lmod software modules extension might also accidently be used to destroy the functionality of the whole JupyterLab instance.
Unloading the \verb|Jupyter/2019a-Python-3.6.8| modules breaks the functionality of the whole set of default Jupyter kernels.
In the software extension module several Jupyter software environments are available

\begin{minted}[breaklines,breakanywhere]{bash}
$ module -t --redirect avail | grep -i jupyter
Jupyter/
Jupyter/2019a-rc17-Python-3.6.8
Jupyter/2019a-rc18-Python-3.6.8
Jupyter/2019a-rc19-Python-3.6.8
Jupyter/2019a-rc20-Python-3.6.8
Jupyter/2019a-rc21-Python-3.6.8
Jupyter/2019a-rc22-Python-3.6.8
Jupyter/2019a-rc23-Python-3.6.8
Jupyter/2019a-rc30-Python-3.6.8
Jupyter/2019a-rc31-Python-3.6.8
Jupyter/2019a-devel-Python-3.6.8
Jupyter/2019a-Python-3.6.8-damian
Jupyter/2019a-Python-3.6.8
\end{minted}

but it remains unclear if stable JupyterLab instance software environments are established.
For example, unloading the \verb|Jupyter/2019a-Python-3.6.8| module does not unload its dependent modules, even though they are activated by the \verb|load()| command, for which Lmod intended behaviour\footnote{see \href{https://lmod.readthedocs.io/en/latest/098_dependent_modules.html}{Lmod 8.3.1 documentation} on module dependencies} is to unload dependencies.
If this a problem on the Lmod side (on JUWELS Lmod 7.7.38 is installed), or by software architecture design is not clear at this point.
SHOULD IT BE INVESTIGATED?

Software environment modules are used to establish and extend the functionality of the instantiated JupyterLab software environment.
However, they are not set up in a strictly hierarchical way, the potential for setting up conflicting software environments is therefore currently high.
Choice of a slightly different design approach might solve that problem, i.e. providing a module hierarchy that only contains Jupyter related modules.

\subsection{Summary}
\label{sect:working-envs:summary}

Here we summarize the status quo of working environment selection, modification and specification for JUWELS JupyterLab instances.
Specific proposals to achieve better robustness and flexibility, compatible with the chosen system software architecture, are also given.
Principal other user working environment managing approaches are discussed at the end of the report in section XXX.

\subsubsection{Selection, modification and specification}

Working environment selection is established based on Jupyter kernels, available per default are currently a Bash, Javascript, Julia and Python kernel, and several C++ kernels.
The default Python package environment can be extended and modified using the Lmod software environment Jupyter extension available in the JupyterLab sidebar, however, most of the available Python modules are not compatible with the default environment, quickly leaving the user with broken Python kernel functionality.
THERE SHOULD BE MORE DETAILS FROM ABOVE INVESTIGATIONS HERE.
The central problem is the entirely missing documentation on what can or should be achieved via the software environment module extension.

Modification of the default Python environment can also be done by setting up a virtual Python environment, a documentation is provided that explains registering it as further Jupyter kernel.
However, the approach that is described only allows the user to modify certain Python packages inside the extensive default Python environment.
Specifying a standalone and light-weight virtual Python environment as Jupyter kernel is possible, but currently not documented.

The specification of completely standalone working environments has been explored in section XXX.
Both a Conda package manager and a container based Jupyter kernel approach are presented.
They extend the documented approaches by the possibility of deploying isolated software environments, enabling convenient analysis environment porting and scientific reproducibility.

\subsubsection{Suggestions}

\begin{itemize}

  \item set up a software modules hierarchy that is scoped only for the JUWELS JupyterLab functionality;
  provided Lmod software environment modules should be completely exchangeable;
  the user should not be able or should clearly be instructed how not to destroy their Jupyter kernels and JupyterLab instances

  \item provide only a minimalistic Python interpreter software environment module for users' to build their own virtual Python environments and Jupyter kernels;
  this would combine the advantage of using a JUWELS specific Python interpreter (with its potentially higher code execution speeds, see  \href{https://www.fz-juelich.de/SharedDocs/Downloads/IAS/JSC/EN/slides/supercomputer-ressources-2019-11/15a-tuning_intel.html}{this example investigation by Intel}) with (robust) standalone pip package dependency managing

  \item extend the documentation for working environment specification based on the alternatives presented in section XXX

\end{itemize}


%\begin{itemize}
%  \item Define own kernels?
%  \item Possible to run remote kernels?
%  \item Use conda-defined kernels?
%\end{itemize}

%\subsection{Existing ways of selecting working environments}
%\label{ss-existing-env-selection}

%\subsection{Desired ways of selecting working environments}
%\label{ss-desired-env-selection}

%\subsection{Existing ways of defining working environments}
%\label{ss-existing-env-definition}

%\subsection{Desired ways of defining working environments}
%\label{ss-desired-env-definition}

%==============================================================================

\section{Availability of resources}
\label{s-availability-resources}

\begin{itemize}
  \item Possible to run many notebook servers per user?
  \item Feedback about availability of resources / waiting time etc.?
  \item Tranparent way to chose resources that are available?
\end{itemize}

%==============================================================================

\section{Stability}
\label{s-stability}

\begin{itemize}
	\item Stable in productive work? Error rate?
  \item Possibility to debug if errors occur?
  \item Responsive under less than ideal link from personal endpoint to Jupyterhub?
\end{itemize}

%==============================================================================

\section{Customization}
\label{s-customization}

\begin{itemize}
	\item Use own notebook server?
  \item Possible to use jupyter-server-proxy?
  \item Jupyter Plugins / Widgets?
\end{itemize}

%==============================================================================

\section{HDF cloud via JSC JupyterHub}
\label{s-hdfcloud-jsc-jhub}

%==============================================================================

\section{Comparison to JupyterHub at DKRZ}
\label{s-comparison-dkrz}

%==============================================================================

\bibliography{references.bib}

\end{document}
