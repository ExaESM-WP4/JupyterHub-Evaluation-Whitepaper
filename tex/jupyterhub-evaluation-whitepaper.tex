\documentclass[11pt,a4paper]{article}

\usepackage[utf8]{inputenc}

\usepackage[strings]{underscore}
\usepackage[round,semicolon]{natbib}
\bibliographystyle{plainnat}

\title{\textbf{JupyterHub@JSC evaluation}}

\author{
  Martin Claus \ % alphabetical order
	Katharina Höflich \\
	Willi Rath}

\begin{document}

\maketitle

%==============================================================================

% STYLE GUIDE
%
% - One sentence per line. This avoids merge conflicts and creates easier to read git diffs.
% - TBC.

%==============================================================================

\section{Introduction}
\label{s-introductoin}

Jupyter notebooks~\citep{Kluyver2016} are increasingly used for scientific work.
Services like Binder~\citep{Jupyter2018}, colab \citep{Google2020, Carneiro2018}, or the Pangeo hubs\citep{robinson2019science}, that are based on public cloud infrastructure, are setting high standards for user experience (ELABORATE ON UX ON BINDER, COLAB, PANGEO?).
There are numerous efforts to establish Jupyter based workflows in HPC centres or on-premise cloud infrastructure:
\begin{itemize}
  \item NCAR runs it on CHEYNENNE\footnote{https://www2.cisl.ucar.edu/resources/computational-systems/cheyenne/software/jupyter-and-ipython},
  \item DKRZ offers a JupyterHub\footnote{https://www.dkrz.de/up/systems/mistral/programming/jupyter-notebook},
  \item and JSC has a solution~\citep{Goebbert2018}.
\end{itemize}

In this paper, we evaluate the JSC JupyterHub with respect to {\em Documentation}, {\em Accessibility}, {\em Flexibility}, {\em Availability}, and {\em Stability}.

%==============================================================================

\section{Documentation from user perspective}
\label{s-doc-from-user-pov}

\begin{itemize}
	\item Are the docs up to date with the system that is offered?
	\item Are the docs up to date with the Jupyter docs?
	\item Is there a way of contributing to the docs?
	\item Is it easy for new Jupyter users to get started just based on what's provided by the centre?
\end{itemize}

%==============================================================================

\section{Working environments}
\label{s-working-envs}

\begin{itemize}
  \item Define own kernels?
  \item Possible to run remote kernels?
  \item Use conda-defined kernels?
\end{itemize}

\subsection{Existing ways of selecting working environments}
\label{ss-existing-env-selection}

\subsection{Desired ways of selecting working environments}
\label{ss-desired-env-selection}

\subsection{Existing ways of defining working environments}
\label{ss-existing-env-definition}

\subsection{Desired ways of defining working environments}
\label{ss-desired-env-definition}

%==============================================================================

\section{Availability of resources}
\label{s-availability-resources}

\begin{itemize}
  \item Possible to run many notebook servers per user?
  \item Feedback about availability of resources / waiting time etc.?
  \item Tranparent way to chose resources that are available?
\end{itemize}

%==============================================================================

\section{Stability}
\label{s-stability}

\begin{itemize}
	\item Stable in productive work? Error rate?
  \item Possibility to debug if errors occur?
  \item Responsive under less than ideal link from personal endpoint to Jupyterhub?
\end{itemize}

%==============================================================================

\section{Customization}
\label{s-customization}

\begin{itemize}
	\item Use own notebook server?
  \item Possible to use jupyter-server-proxy?
  \item Jupyter Plugins / Widgets?
\end{itemize}

%==============================================================================

\section{HDF cloud via JSC JupyterHub}
\label{s-hdfcloud-jsc-jhub}

%==============================================================================

\section{Comparison to JupyterHub at DKRZ}
\label{s-comparison-dkrz}

%==============================================================================

\bibliography{references.bib}

\end{document}
