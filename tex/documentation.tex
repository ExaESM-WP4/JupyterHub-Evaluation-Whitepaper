
\subsection{Documentation from user perspective}
\label{s-doc-from-user-pov}

On the central page of the Jupyter@JSC hub there are easy to reach links to some documentation, but there is no condensed material suitable for users that need/expect a quick overview on how and why to use the JSC JupyterHub.
There is different places\footnote{Gitlab@JSC, Visualization group wiki: https://trac.version.fz-juelich.de/vis/ (via https://www.fz-juelich.de/ias/jsc/EN/Expertise/Support/Visualization/_node.html and https://www.fz-juelich.de/ias/jsc/EN/Expertise/Support/Visualization/InteractiveSupercomputing/artikel.html?nn=1538466), FZJ-JSC@Github, ...} that contain (often heavily outdated) examples, tutorials, and other documentation related to Jupyter@JSC.
These are, however, not discoverable\footnote{2020-06-24} directly from the central JupyterHub login/control panel page.

Another way of learning about Jupyter@JSC are the training events provided by JSC on a regular basis\footnote{put link to website here}.
There was a dedicated training course for Jupyter within the PRACE network scheduled, that needed to be postponed because of the Corona outbreak of 2020\footnote{link here}.

The current documentation covers customization\footnote{}, JSC specifics\footnote{} and other aspects that are predominantly interesting for expert (Jupyter) users, but there is no easy to use introductory documentation suitable for beginners.
The expert guides often are not in sync with the actual state of the hub, or do not work for other reasons\footnote{see e.g. section \ref{tech-details-appendix} for an impression on proposed and actually working ways of customizing Jupyter kernels}.

While there are pointers to more general community documentation on the hub control page, these are not curated in a way that really help JSC users to find useful existing community documentation.
As stated above there is many outdated examples that reflect the state of the JSC hub and the state of the underlying community software at the time the materials were created.
Many of those materials won't work with the current state of the JSC hub and of with the current stable versions of the highlighted community software packages\footnote{see e.g. Dask materials which are based on workshop materials which have been developed for Dask < v1.0, while the current stable Dask release is v2.19.0}.
Within a JupyterLab session on the JSC HPC systems there is another set of example materials\footnote{https://gitlab.version.fz-juelich.de/jupyter4jsc/j4j_notebooks/}.
These are directly executable on the machines, and appear to be actively maintained\footnote{see https://gitlab.version.fz-juelich.de/jupyter4jsc/j4j_notebooks/-/commits/}.
For JSC users it is in principle possible to contribute to the JSC-specific documentation via the Git repository containing the examples.
Users are, however, not actively motivated to do so and from the Git repository\footnote{https://gitlab.version.fz-juelich.de/jupyter4jsc/j4j_notebooks/-/issues} there is no indication that there have been any user contributions so far.

JupyterHub@JSC appears to be under constant development\footnote{https://github.com/FZJ-JSC?q=jupyter}.
For users that already use the system, there is no easy way of getting an overview of functional changes.
