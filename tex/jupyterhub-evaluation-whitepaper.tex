\documentclass[11pt,a4paper]{article}

\usepackage{xcolor}
\definecolor{bg}{rgb}{0.96,0.96,0.96}
\usepackage[cache=false]{minted}
\setminted[]{fontsize=\small,bgcolor=bg}

\usepackage{hyperref}

\usepackage[utf8]{inputenc}
\usepackage{indentfirst}

\usepackage[strings]{underscore}
\usepackage[round,semicolon]{natbib}
\bibliographystyle{plainnat}

\usepackage{cprotect}  % allows for verbatim text in footnotes

\title{\textbf{Jupyter services at high-performance computing centers from a scientific target user perpective}}

\author{
	Katharina Höflich\\
	Willi Rath\\
  Martin Claus}

\begin{document}

\maketitle
\begin{abstract}
Several HPC system operators have started to provide web-based access solutions to HPC compute/storage infrastructure.
Implementation options are immense and system providers need to consider and weigh aspects such as implementation effort / cost, security and maintainability against target user service usability.
Thus, services might not turn out to be optimal in supporting user productivity.
Here, scientific user producivity/usability requirements are formulated and existing web-based access solutions to HPC resources are evaluated against these requirements.
Suggestions are given that are targeted not only at maximizing the productivity of scientific target users, but also at minimizing system provider/administator workload/manual intervention.
This white paper aims at closing a gap in ... [final words here].
\end{abstract}


\tableofcontents

%==============================================================================

% STYLE GUIDE
%
% - One sentence per line. This avoids merge conflicts and creates easier to read git diffs.
% - TBC.

%==============================================================================


\section{Introduction}
\label{s-introductoin}

Jupyter notebooks~\citep{Kluyver2016} are increasingly used for scientific work.
Services like Binder~\citep{Jupyter2018}, colab \citep{Google2020, Carneiro2018}, or the Pangeo hubs\citep{robinson2019science}, that are based on public cloud infrastructure, are setting high standards for user experience (ELABORATE ON UX ON BINDER, COLAB, PANGEO?).
There are numerous efforts to establish Jupyter based workflows in HPC centres or on-premise cloud infrastructure:
\begin{itemize}
  \item NCAR runs it on CHEYNENNE\footnote{https://www2.cisl.ucar.edu/resources/computational-systems/cheyenne/software/jupyter-and-ipython},
  \item DKRZ offers a JupyterHub\footnote{https://www.dkrz.de/up/systems/mistral/programming/jupyter-notebook},
  \item and JSC has a solution~\citep{Goebbert2018}.
\end{itemize}

In this paper, we evaluate the JSC JupyterHub with respect to {\em Documentation}, {\em Accessibility}, {\em Flexibility}, {\em Availability}, and {\em Stability}.


%==============================================================================

\section{JupyterHub at JSC service}

\subsection{HPC system infrastructure}

(Might be helpful for the discussion to have a basic overview on the JSC hardware/software infrastructure. Not everything can, but other things can only be accessed via the JupyterHub service.)


\subsection{Resources available from the JupyterHub}

% Highlight how the resources available via classical HPC jobs can be used from the JupyterHub



\subsection{Documentation from user perspective}
\label{s-doc-from-user-pov}

On the central page of the Jupyter@JSC hub there are easy to reach links to some documentation, but there is no condensed material suitable for users that need/expect a quick overview on how and why to use the JSC JupyterHub.
There is different places\footnote{Gitlab@JSC, Visualization group wiki, FZJ-JSC@Github, ...} that contain (often heavily outdated) examples, tutorials, and other documentation related to Jupyter@JSC.
These are, however, not discoverable\footnote{2020-06-24} directly from the central JupyterHub login/control panel page.

The current documentation covers customization\footnote{}, JSC specifics\footnote{} and other aspects that are predominantly interesting for expert (Jupyter) users, but there is no easy to use introductory documentation suitable for beginners.
The expert guides often are not in sync with the actual state of the hub, or do not work for other reasons\footnote{see e.g. section \ref{tech-details-appendix} for an impression on proposed and actually working ways of customizing Jupyter kernels}.

While there are pointers to more general community documentation on the hub control page, these are not curated in a way that really help JSC users to find useful existing community documentation.
As stated above there is many outdated examples that reflect the state of the JSC hub and the state of the underlying community software at the time the materials were created.
Many of those materials won't work with the current state of the JSC hub and of with the current stable versions of the highlighted community software packages\footnote{see e.g. Dask materials which are based on workshop materials which have been developed for Dask < v1.0, while the current stable Dask release is v2.19.0}.
Within a JupyterLab session on the JSC HPC systems there is another set of example materials\footnote{https://gitlab.version.fz-juelich.de/jupyter4jsc/j4j_notebooks/}.
These are directly executable on the machines, and appear to be actively maintained\footnote{see https://gitlab.version.fz-juelich.de/jupyter4jsc/j4j_notebooks/-/commits/}.
For JSC users it is in principle possible to contribute to the JSC-specific documentation via the Git repository containing the examples.
Users are, however, not actively motivated to do so and from the Git repository\footnote{https://gitlab.version.fz-juelich.de/jupyter4jsc/j4j_notebooks/-/issues} there is no indication that there have been any user contributions so far.

JupyterHub@JSC appears to be under constant development\footnote{https://github.com/FZJ-JSC?q=jupyter}.
For users that already use the system, there is no easy way of getting an overview of functional changes.



\subsection{Working environments}

TODO: Somewhere here mention stability and reliability.
\begin{itemize}
	\item Stable in productive work? Error rate?
  \item Possibility to debug if errors occur?
  \item Responsive under less than ideal link from personal endpoint to Jupyterhub?
\end{itemize}

Here we summarize the status quo of working environment selection, modification and specification for JUWELS JupyterLab instances and give specific suggestions to achieve better robustness and flexibility, compatible with the chosen system software architecture.

A detailed description of the technical details of the JUWELS Jupyter setup is given in the appendix~\ref{app:tech-details-jsc-hub} with suggestions for simpler implementations of user-defined kernels (see~\ref{sect:custom-kernels}) based on Conda and containers are given.

Principal other user working environment managing approaches are discussed at the end of the report. OR SHOULD WE NOT DO THIS? (JupyterLab instances could also be started inside containers. Currently, this approach is used only on the HDF cloud. Should this be made possible on JURECA and JUWELS, too? What is pro and contra here? From a user and system administrator perspective?)

\subsubsection{Documented functionality for selection, modification and specification of working environments}

At JSC, working environment selection is provided via Jupyter kernels.
Per default, tehre are currently a Bash, Javascript, Julia and Python kernel, and several C++ kernels available.
The default Python package environment can be extended and modified using the Lmod software environment Jupyter extension available in the JupyterLab sidebar.

There is a range of problems with the module based approach of letting the user modify their environment.
First, there is no documentation of typical usage patterns and anti-patterns and no information on what the user can or cannot expect to achieve via the software environment module extension.
Then, most of the available Python modules are not compatible with the default environment, and users are esily getting to a state of broken Python kernels or even a defunct Jupyter instance (see~\ref{sect:lmod-use-and-problems}).
Finally, the documented ways of creating custom Jupyter kernels with a virtual Python environment hardly goes beyond slight modifications of an extensive default Python environment.

\subsubsection{Undocumented ways of specifying working environments}

Specifying a more standalone and light-weight virtual Python environment as Jupyter kernel is currently not documented but can easily be done (see~\ref{sect:suggested-python-virtual-environment}).
The specification of completely standalone working environments has been explored in sections \ref{sect:conda-environment} and \ref{sect:container-based-environment}.
Both a Conda package manager and a container based Jupyter kernel approach are suggested.
They extend the documented approaches by the possibility of deploying isolated software environments, enabling convenient porting of analysis environments and facilitating scientific reproducibility.

\subsubsection{Further customization}

\begin{itemize}
  \item Use own notebook server?
  \item Possible to use jupyter-server-proxy?
  \item Jupyter Plugins / Widgets?
\end{itemize}

\subsubsection{Suggestions}

\begin{itemize}

  \item Set up a software modules hierarchy that is scoped only for the JUWELS JupyterLab functionality and ensures the provided Lmod software environment modules are completely exchangeable.
  The user should not have a way of or should be aware of how to avoid  destroying their Jupyter kernels and JupyterLab instances.

  \item On system start-up, the \verb|Jupyter/2019a-Python-3.6.8| module automatically establishes the software environment for all available Jupyter kernels.
  To prevent version conflicts, it might make sense to provide only a very minimalistic Jupyter environment and prevent the user from unloading the modules that provide basic system functionality.
  Loading environment modules that provide Jupyter kernel functionality should be documented in a way the lets users consciously manage their software environments.
  This could, however, need modifications of the Jupyter kernel launcher tab, because currently, the launcher tab only shows the Jupyter kernels available on JupyterLab instance start-up and kernels that become available after JupyterLab has started can only be chosen from the kernel drop down of already open Jupyter notebooks.

  \item Another approach to create robust Jupyter kernels would be to make module-based kernel launcher scripts the default, let these show up in the JupyterLab launcher tab on system start-up.
  Loading and unloading of software environment modules would not have an effect on the kernel software environments and customization of kernel software environments would be done entirely by setting up own kernel launcher scripts.
  In this case good documentation would need to be provided by the system designers/administators and the Lmod software modules extension in the JupyterLab sidebar would become obsolete.

  \item Provide only a minimalistic Python interpreter software environment module for users to build their own virtual Python environments and Jupyter kernels.
  This would combine the advantage of using a JUWELS specific Python interpreter (with its potentially higher code execution speeds, see \href{https://www.fz-juelich.de/SharedDocs/Downloads/IAS/JSC/EN/slides/supercomputer-ressources-2019-11/15a-tuning_intel.html}{this example investigation by Intel}) with robust standalone pip package dependency managing.

  \item Extend the documentation for working environment specification based on the alternatives presented in section \ref{sect:jupyter-kernel-suggestions}.

  \item It should be noted that a complete and clear separation of JupyterLab and kernel functionality is not possible with the Jupyter ecosystem.
  Hence, a one-size-fits-all JupyterLab configuration that seamlessly works with every conceivable kernel cannot be provided.
  A possible way out would be to allow users to not only modify or create kernels, but to allow for full configuration also of upyterLab.
  This could be done with containers (which then would not only contain a Jupyter kernel but the full JupyterLab and kernel application) or by letting users specify virtual or Conda environments that also provide the Jupyter instance that is spawned by the JupyterHub.

\end{itemize}




TODO: Somewhere here mention stability and reliability.
\begin{itemize}
	\item Stable in productive work? Error rate?
  \item Possibility to debug if errors occur?
  \item Responsive under less than ideal link from personal endpoint to Jupyterhub?
\end{itemize}

%==============================================================================

\section{Alternative: Jupyter@HPC without a hub}

Scientists from GEOMAR have been using Jupyter on HPC systems for many years without relying on a JupyterHub that is maintained by system providers.
From 2016 to 2018, GEOMAR staff maintained a JupyterHub that used the \verb|remote_ikernel|\footnote{https://github.com/tdaff/remote_ikernel} package for spawning kernels on remote hosts and had a mechanism of using SSH keys for authentication without the need to collect keys on an untrusted shared machine.
It provided semi-automatic ways of adding centrally managed default kernels as well as arbitrary user-defined kernels (that needed to be installed on the HPC centres, however).

In 2018, this centrally managed JupyterHub was decommissioned in favor of an approach that does not provide \emph{any} infrastructure and purely relies on enabling and educating users in maintaining their own Jupyter installations.
Documetation can be found on \url{https://git.geomar.de/python/jupyter_on_HPC_setup_guide/}.

The main aspects of this documentation based approach are
\begin{itemize}
    \item a setup guide for conda-based Python environments on any Unix-like operating system,
    \item a guide for managing Python kernels as Conda environments, and
    \item a guide for using SSH-based socks proxies for connecting to any HTTP service running on a remote host that comes with a wrapper script automating tunnel creation and proxy setup on Linux, Mac OSX and Windows.
\end{itemize}

This documentation-based approach has been a full success in that it almost completely eliminated the need of constant support by central technical staff.
By making users of from levels of experience aware of all the interacting parts of what they do on the HPC centre, it helped them debug problems themselves or, by building common knowledge, to support each other.
By providing a way of contributing to the documentation, the workload for the technical staff could be minimized as well.
The only aspect that needed some attention from GEOMAR technical staff is the automated SSH tunnel-based proxied browser which had to be adapted to new security functionality of Chromium approximately once per year.

From a user perspective, the purely documentation-based approach adds limited effort for getting started\footnote{Working through 2-3 pages of documentation rather than logging in and starting immediately.}, but brings substantial intermediate and long-term advantages, because all users learn is completely system-agnostic.
In fact, users familiar with setting up their own Jupyter-based workflows often decided against working with a JuptyerHub even if it was provided, because of the substantial overhead for adapting their own easily portable workflows and environments to a system with limited configurability.


%==============================================================================

\section{Jupyter at DKRZ}
\label{sect:jupyter-at-dkrz}

In the following, the Jupyter-based solutions provided by DKRZ\footnote{Deutsches Klimarechenzentrum} are investigated.
First, a brief overview on the available compute infrastructure is given, then the officially supported Jupyter-based access options are described.
The goal is to enable a comprehensive discussion of the JupyterHub@JSC from a more general Jupyter@HPC viewpoint.

\subsection{HPC system infrastructure}

DKRZ operates \href{https://www.dkrz.de/up/systems/mistral}{Mistral} (HLRE-3) which is a tier-2 HPC system \cite{Wissenschaftsrat2015, GaussAllianz2020} with Earth system researchers as target user group.
The system was originally installed in July 2015\footnote{https://www.dkrz.de/up/systems/mistral/configuration} and is planned to operate until mid-2021\footnote{https://www.dkrz.de/kommunikation/aktuelles/dkrz-verfuenffacht-supercomputing-leistung-mit-neuem-bullsequana-von-atos}.
The system currently consists of roughly 1600 Intel Xeon E5-2680v3 phase-1 and 1800 Intel Xeon E5-2695V4 phase-2 compute nodes.
A very small fraction, i.e. 21 compute nodes (about 0.6\% of the total system, available via the \verb|gpu| partition) additionally operates a pair of several different types of Nvidia Tesla GPUs\footnote{what types of GPUs?}.
The main part of the system, i.e. about 96.2\% of the HPC system resources are reserved for classic multi- and full-node batch compute tasks (\verb|compute| and \verb|compute2| partition), with only a very small fraction of 1.1\% being reserved for shared-node compute tasks (\verb|shared| partition), respectively.
Another about 1.3\% of the system is explicitely reserved for data processing tasks (\verb|prepost| partition), whereas the remaining about 0.9\% are reserved for XXX tasks (\verb|miklip| partition)\footnote{are these dedicated for a special compute project?}.
The system is supplemented by seven login nodes which are dedicated for file editing, source code compilation, and the preparation and monitoring of batch tasks\footnote{https://www.dkrz.de/up/systems/mistral/login-and-environment} only.
Interactive command line analysis tasks are supposed to happen on the five \verb|mistralpp| interactive nodes\footnote{add URL}, that can also directly be accessed via external SSH sessions.

\subsection{Documentation from user perspective}

Documentation about how to do general interactive and batch-type data analysis/processing tasks are provided only on and are easily found via the DKRZ user portal \footnote{https://www.dkrz.de/up/services/analysis/data-processing/processing-on-mistral}.
The documentation about Jupyter is also centrally managed and easy to find on the user portal\footnote{}.
There is, however, no obvious way of getting to the documentation from the actual JupyterHub login page and/or control panel.

The DKRZ documents two supported ways of working with Jupyter on Mistral.
They offer a short "how to" and a bash script that allows starting standalone Jupyter instances on compute nodes and they offer a JupyterHub service.
Both ways of working with Jupyter are documented separately.
The documentation of the script approach lives in the Mistral specific documentation\footnote{https://www.dkrz.de/up/systems/mistral/programming/jupyter-notebook}, while the JupyterHub is documented as a separate service\footnote{https://www.dkrz.de/up/systems/jupyterhub-dkrz.de-1}.

The JupyterHub documentation does not discriminate between differente experience levels, but is very concise and makes it easy to find relevant information for all Jupyter-experience levels of users.
There is a walkthrough documentation for beginning users demonstrates every step from login to actual work on the HPC cluster, including recommendations about which of the available job profiles (see section \ref{} XXX below) are suitable for which task.
It especially highlights machine specific and SLURM scheduler specific background information that is helpful for unexperienced users and for experienced users not familar with HPC systems.
Where possible, links to the community documentation of Jupyter are given\footnote{as the link to the community documentation points to the latest version, there is a possible source of inconsistency/confusion, though}.
A target user group specific example Jupyter notebook is linked and available via the DKRZ Gitlab server.

The documentation seems to be in good agreement with the actually deployed Jupyter service.
It should be noted though that the deployed Jupyter version is approximately two years old\footnote{version 0.7.2 released in XXX}.

The given examples seem to match the actually deployed Jupyter service.
There doesn't seem to have been a lot of changes to the JupyterHub system since its initial deployment, thus it's not possible to say how the documentation of changes to the JupyterHub system is done/solved.
Currently, there is no explicit version information on both the documentation and the examples.

\subsection{JupyterHub service}

\subsection{Jupyter remote control script}

At DKRZ a bash script\footnote{\url{https://gitlab.dkrz.de/k202009/ssh_scripts/-/blob/35560ae0c72d9d99e1448140ec0ac2210f7fd1e5/start-jupyter}} that allows users to remotely manage their Jupyter processes on Mistral is provided.
The script is executed on the user's personal computer and in a typical working session will
\begin{itemize}
    \item establish an SSH session to Mistral
    \item use a Jupyter environment definition stored on Mistral
    \item submit a Jupyter server either on mistralpp or as a batch job
    \item waits for the batch job to start (if necessary)
    \item setups an SSH tunnel
    \item opens the local web browser
    \item connects to remote Jupyter server via tunnel
\end{itemize}

The script is designed to work on Linux, Mac and Windows 10 systems.
While it is designed to provide a solution to setting up, connecting and terminating a remote Jupyter server, it doesn't provide a robust way of re-established a disrupted connection.
Jupyter servers need to be terminated manually, which might lead to orphaned Jupyter sessions both on mistralpp and the batch scheduler nodes.
The documentation highlights this problem for the batch scheduler nodes and gives an example on how to terminate an orphaned job\footnote{https://www.dkrz.de/up/systems/mistral/programming/jupyter-notebook}.
(Is this a problem also for mistralpp?)

Compared to the DKRZ JupyterHub the management of Jupyter working environments is very flexible, because the only requirement for the Jupyter server that is started on Mistral is that an access URL is printed to the processes' standard output.
In addition, the script can be modified and then allows for arbitrary resource requests on Mistral, i.e. everything that could be requested via the command line, and hence gives more flexibility than the pre-defined job profiles provided by the drop-down menu in their JupyterHub solution.


%==============================================================================

\section{Discussion}

(Here, discuss the JSC solution with respect to the user requirements and key challenges stated in the introduction. Also incorporate aspects from the other investigated solutions.)

\begin{itemize}
  \item do we need a JupyterHub? for which task, and for which not?
  \item ...
\end{itemize}


%==============================================================================

\section{Recommendations for Jupyter@JSC}

\subsection{HPC resources}

\begin{itemize}
  \item \emph{Provide information on queue waiting times:} We recommend Jupyter@JSC provides a way of estimating possible waiting times, e.g., by generating a graphical representation of the current system occupation.
  This could greatly enhance the way users can plan their interactive work.
\end{itemize}

\subsection{Working environments}

\subsubsection{Pre-installed software}

\begin{itemize}
  \item \emph{Make optimized libraries easy to re-use:} For system specific low-level software that is critical for performance, we recommend the Jupyter@JSC system provides easy to use builds in a way that enables users in bringing / porting their own software selection based on these. E.g., if there is system-optimized builds of modules like numpy or cuda, make it easy to use them in own environments without forcing users to use other parts of the default envs.
  \item \emph{Remove or restrict the Lmod plugin:} We recommend, the Lmod plugin is removed or to keep restricted to only provide the Jupyter-specific part of the module tree.
  \item \emph{Actively support pip and conda:} To accomodate the users who come from the Python data analysis world, there should be an easy way of using the community standards (mainly pip and conda) for defining working environments.
\end{itemize}

\subsubsection{Configurability of working environments}

As was argued above, the separation between the Jupyter server and the Jupyter kernel is far from perfect.
This can result in conflicts between kernel and Jupyter server if, e.g., for reproducibility reasons, users keep un-changed kernels for a long time while the default JupyterLab setup is evolving, or if users use the full configurability of the kernels and create conflicts with the current default JupyterLab.

\begin{itemize}
  \item \emph{Make Jupyter user servers fully configurable:} To overcome this conflict, we recommend that Jupyter@JSC provides full configurability of not only the Jupyter kernels but also of the Jupyter user server.
  \item \emph{Allow for spawning old configurations of the defaults:} To also accomodate users with reproducibility demands but who do not intend to completely manage their own environments, we recommend Jupyter@JSC makes available frozen or checkpointed versions of the default Jupyter environments to be spawned via the hub.
\end{itemize}

\subsubsection{Conrtainer-based Jupyter and kernels}

\begin{itemize}
  \item \emph{Allow for container-based kernels:} Containers provide a way of isolating and porting working environments.
  Hence, they can be used for defining kernels (see \ref{sect:container-based-environment}) that are fully isolated from pre-defined environments.
  \item \emph{Allow for fully container-based Jupyter user servers:} There is ways of building user-defined full working environment including the Jupyter user server\footnote{CITE BINDER / LINK TO repo2docker} that are supported by the Jupyter community.
  These could solve virtually all of the above recommendations.
\end{itemize}

\subsection{Documentation}

\begin{itemize}
  \item \emph{Consolidate documentation:} Make available the system-specific documentation in a way that does not need a login to the system, and provide an obvious way for users to contributed to the JSC specific docs.
  \item \emph{Leverage community documentation:} Refer to existing community documentation whenever / wherever possible rather than repeating or copying non-synchronized versions of central materials.
  \item completely removes or clearly marks as obsolete any outdated JSC specific docs documentation,
  \item \emph{Document evolution of Jupyter@JSC:} Provides a changelog on Jupyter@JSC implementation to guide existing users who need a quick overview of new possibilities or restrictions and remove remove or mark as obsolete any outdated JSC specific documentation.
  \item \emph{Document intended use:} Provide documentation about usage patterns / anti-patterns (e.g., when to use / not use login nodes / batch nodes).
  \item \emph{Document resource limitations:} Make sure users understand where / when CPU hours are billed, and clearly highlight possible resource limitations (e.g., walltime or memory limits for Jupyter spawned on login nodes).
  \item \emph{Highlight system-specific optimizations:} Provide documentation on where to find, and how and why to use system-optimized performance-critical libraries and modules that is able to inform users who bring or build their own environments.
\end{itemize}


%==============================================================================

\section{Outlook}

(Are there any implications for the design of HPC infrastructure? Maybe treat these aspects in the discussion.)

\begin{itemize}
  \item Hardware infrastructure
  \item Software/OS configuration
  \item Queue/System node design
  \item Added cloud-like resources right next to the HPC machine (like HDFCoud)
\end{itemize}


%==============================================================================

\section{Acknowledgements}

Katja Matthes and Sebastian Wahl for a compute project membership at DKRZ (NATHAN: Quantification of Natural Climate Variability in the Atmosphere-Hydrosphere System with Data Constrained Simulations) which allowed to investigate the DKRZ Jupyter solutions in the required detail.

%==============================================================================

\appendix


\section{Technical details of the JSC JupyterHub setup}
\label{sect:tech-details-jsc-hub}

In this section, an in-depth description of the software environments accessible via the JupyterHub system at JSC is given.

\subsection{JupyterLab system description}
\label{sect:system-description}

JupyterLab sessions on JUWELS nodes are instantiated based on activating the following hierarchy of \href{https://lmod.readthedocs.io/en/latest/index.html}{Lmod software environment modules}:
%
\begin{minted}[breaklines,breakanywhere]{bash}
$ module purge --force
$ module use $OTHERSTAGES
$ module load Stages/Devel-2019a GCC/8.3.0
$ module load Jupyter/2019a-Python-3.6.8
$ which jupyter
/gpfs/software/juwels/stages/Devel-2019a/software/Jupyter/2019a-gcccoremkl-8.3.0-2019.3.199-Python-3.6.8/bin/jupyter
\end{minted}

Executing the following command on JUWELS login nodes starts up the same JupyterLab instance that is otherwise spawned by the JupyterHub system:
%
\begin{minted}[breaklines,breakanywhere]{bash}
$ jupyter lab --ip=$HOSTNAME --no-browser
\end{minted}

The following software environment modules\footnote{the command must be executed before the Jupyter module is loaded} are further activated by the \verb|Jupyter/2019a-Python-3.6.8|:
%
\begin{minted}[breaklines,breakanywhere]{bash}
$ module --redirect show Jupyter/2019a-Python-3.6.8 | grep "load("
load("imkl/.2019.3.199")
load("Python/3.6.8")
load("SciPy-Stack/2019a-Python-3.6.8")
load("libyaml/.0.2.2")
load("cling/.0.6dev")
load("pandoc/2.7.2")
load("texlive/2018")
load("Julia/1.1.0")
load("ITK/5.0.1-Python-3.6.8")
load("HDF5/1.10.5-serial")
load("netcdf4-python/1.5.3-serial-Python-3.6.8")
load("FFmpeg/.4.1.3")
\end{minted}
%
They provide the default software environment that is used by the following Jupyter kernels
%
\begin{minted}[breaklines,breakanywhere]{bash}
$ jupyter kernelspec list
Available kernels:
  bash          /gpfs/software/juwels/stages/Devel-2019a/software/Jupyter/2019a-gcccoremkl-8.3.0-2019.3.199-Python-3.6.8/share/jupyter/kernels/bash
  cling-cpp11   /gpfs/software/juwels/stages/Devel-2019a/software/Jupyter/2019a-gcccoremkl-8.3.0-2019.3.199-Python-3.6.8/share/jupyter/kernels/cling-cpp11
  cling-cpp14   /gpfs/software/juwels/stages/Devel-2019a/software/Jupyter/2019a-gcccoremkl-8.3.0-2019.3.199-Python-3.6.8/share/jupyter/kernels/cling-cpp14
  cling-cpp17   /gpfs/software/juwels/stages/Devel-2019a/software/Jupyter/2019a-gcccoremkl-8.3.0-2019.3.199-Python-3.6.8/share/jupyter/kernels/cling-cpp17
  javascript    /gpfs/software/juwels/stages/Devel-2019a/software/Jupyter/2019a-gcccoremkl-8.3.0-2019.3.199-Python-3.6.8/share/jupyter/kernels/javascript
  julia-1.1     /gpfs/software/juwels/stages/Devel-2019a/software/Jupyter/2019a-gcccoremkl-8.3.0-2019.3.199-Python-3.6.8/share/jupyter/kernels/julia-1.1
  python3       /gpfs/software/juwels/stages/Devel-2019a/software/Jupyter/2019a-gcccoremkl-8.3.0-2019.3.199-Python-3.6.8/share/jupyter/kernels/python3
\end{minted}
%
and that can be chosen in the JupyterLab launcher tab.

\subsection{The Lmod software Juptyer extension}

The default JupyterLab instance software environment can directly be modified with the \href{https://github.com/cmd-ntrf/jupyter-lmod}{Jupyter Lmod extension} that is accessible via the JupyterLab sidebar.
While the idea of using the well-established module concept for fine-grained managing of Python environments in Jupyter is appealing, the current Lmod and module setup is not strictly hierarchical and hence brings a lot of potential for creating inconsistent states that unintended versions or even broken environments that are very difficult to debug from within the JupyterLab session.

I DON'T UNDERSTAND THIS: Choice of a slightly different design approach might solve that problem, i.e. providing a module hierarchy that only contains Jupyter related modules.

\subsubsection{Intended behaviour of the Lmod software Jupyter extension}

The currently activated default software environment modules are listed, as well as every module available in the \verb|Devel-2019a| stage and \verb|GCC/8.3.0| compiler module hierarchy\cprotect\footnote{These modules also listed by the \verb|module avail| command in a shell.}.

Using the Lmod softwares extensions approach, the default Python environment can, for example, be extended.
Adding, e.g., the \verb|Keras/2.2.4-Python-3.6.8| software module\footnote{also available in a GPU variant} automatically sets up the necessary \verb|TensorFlow/1.13.1-Python-3.6.8| backend and applies the following changes to the environmen\footnote{To make these changes available in the Jupyter kernel, the kernel needs to be restarted.}:
%
\begin{minted}{bash}
$ pip list > default.txt
$ module load Keras/2.2.4-Python-3.6.8
$ pip list > changed.txt
$ diff default.txt changed.txt
3c3
< absl-py                            0.8.1
---
> absl-py                            0.7.1
14a15
> astor                              0.7.1
72a74
> gast                               0.2.2
76a79
> grpcio                             1.20.1
144a148,150
> Keras                              2.2.4
> Keras-Applications                 1.0.7
> Keras-Preprocessing                1.0.9
153a160
> Markdown                           3.1
277a285,288
> tensorboard                        1.13.1
> tensorflow                         1.13.1
> tensorflow-estimator               1.13.0
> termcolor                          1.1.0
\end{minted}

\subsubsection{Creating inconsistent environments with Lmod}

While for the previous example applied software environment changes are successful, conflicting examples can also be identified.
Loading, e.g., the \verb|numba/0.43.1-Python-3.6.8| module only activates an older version of the LLVM compiler infrastructure and the llvmlite Python package. The version of the numba package that is present in the Jupyter kernel, however, does not change:
%
\begin{minted}{bash}
$ module load numba/0.43.1-Python-3.6.8
The following have been reloaded with a version change:
  1) LLVM/8.0.0 => LLVM/7.0.1-dev
$ pip list > changed.txt
$ diff default.txt changed.txt
149c149
< llvmlite                           0.30.0
---
> llvmlite                           0.28.0
$ pip show numba | grep -i version
Version: 0.46.0
\end{minted}

Evaluation of the Python package environment reveals that numba from the \verb|Jupyter/2019a-Python-3.6.8| environment is used, even though the \verb|PYTHONPATH| is correctly set:
%
\begin{minted}[breaklines,breakanywhere]{bash}
$ module --redirect show numba/0.43.1-Python-3.6.8 | grep PYTHONPATH
prepend_path("PYTHONPATH","/gpfs/software/juwels/stages/Devel-2019a/software/numba/0.43.1-gcccoremkl-8.3.0-2019.3.199-Python-3.6.8/lib/python3.6/site-packages")
$ pip show numba | grep -i location
Location: /gpfs/software/juwels/stages/Devel-2019a/software/Jupyter/2019a-gcccoremkl-8.3.0-2019.3.199-Python-3.6.8/lib/python3.6/site-packages
$ printenv | grep -i pythonpath
PYTHONPATH=/gpfs/software/juwels/stages/Devel-2019a/software/numba/0.43.1-gcccoremkl-8.3.0-2019.3.199-Python-3.6.8/lib/python3.6/site-packages:/gpfs/software/juwels/stages/Devel-2019a/software/Jupyter/2019a-gcccoremkl-8.3.0-2019.3.199-Python-3.6.8/lib/python3.6/site-packages:/gpfs/software/juwels/stages/Devel-2019a/software/netcdf4-python/1.5.3-gcccoremkl-8.3.0-2019.3.199-serial-Python-3.6.8/lib/python3.6/site-packages:/gpfs/software/juwels/stages/Devel-2019a/software/ITK/5.0.1-gcccoremkl-8.3.0-2019.3.199-Python-3.6.8/lib/python3.6/site-packages:/gpfs/software/juwels/stages/Devel-2019a/software/VTK/8.2.0-gcccoremkl-8.3.0-2019.3.199-Python-3.6.8/lib64/python3.6/site-packages:/gpfs/software/juwels/stages/Devel-2019a/software/SciPy-Stack/2019a-gcccoremkl-8.3.0-2019.3.199-Python-3.6.8/lib/python3.6/site-packages:/gpfs/software/juwels/stages/Devel-2019a/software/Python/3.6.8-GCCcore-8.3.0/easybuild/python:/gpfs/software/juwels/stages/Devel-2019a/software/Python/3.6.8-GCCcore-8.3.0/lib/python3.6/site-packages
$ ls /gpfs/software/juwels/stages/Devel-2019a/software/Jupyter/2019a-gcccoremkl-8.3.0-2019.3.199-Python-3.6.8/lib/python3.6/site-packages -lrtha | grep -i numba
drwxr-sr-x  24 goebbert1 swmanage  32K 25. Jan 09:24 numba
drwxr-sr-x   2 goebbert1 swmanage 4,0K 25. Jan 09:24 numba-0.46.0.dist-info
$ ls /gpfs/software/juwels/stages/Devel-2019a/software/numba/0.43.1-gcccoremkl-8.3.0-2019.3.199-Python-3.6.8/lib/python3.6/site-packages -lrtha | grep -i numba
drwxrwsr-x 4 swmanage swmanage 4,0K  9. Mai 2019  numba-0.43.1-py3.6-linux-x86_64.egg
\end{minted}

Generally, such unexpected software module environment behaviour is not desired and could lead to conflicting software with broken Jupyter kernel functionality.

\subsubsection{Breaking kernels and JupyterLab with Lmod}

By using the Lmod software modules extension users can completely destroy their default Jupyter kernel\footnote{For the current session.}, and even the disable the whole JupyterLab instance.

The Lmod softwares Jupyter extension allows, for example, replacing the default \verb|Julia/1.1.0| with a newer \verb|Julia/1.3.1| software environment.
This, however, breaks Julia kernel functionality.
Checking the Jupyter kernel configuration file reveals that the Julia path is hard coded to the 1.1.0 version.
%
\begin{minted}[breaklines,breakanywhere]{bash}
$ cd /gpfs/software/juwels/stages/Devel-2019a/software/Jupyter/2019a-gcccoremkl-8.3.0-2019.3.199-Python-3.6.8/share/jupyter/kernels/julia-1.1
$ cat kernel.json
{
  "display_name": "Julia 1.1.0",
  "argv": [
    "/gpfs/software/juwels/stages/Devel-2019a/software/Julia/1.1.0-gcccoremkl-8.3.0-2019.3.199/bin/julia",
    "-i",
    "--startup-file=yes",
    "--color=yes",
    "--project=@.",
    "/gpfs/software/juwels/stages/Devel-2019a/software/Jupyter/2019a-gcccoremkl-8.3.0-2019.3.199-Python-3.6.8/share/julia/site/packages/IJulia/F1GUo/src/kernel.jl",
    "{connection_file}"
  ],
  "language": "julia",
  "env": {},
  "interrupt_mode": "signal"
}
\end{minted}
%
Replacing the software environment paths with those of Julia 1.3.1, but calling Julia 1.1.0 apparently leads to conflicts that then cause a dead kernel.
The JupyterLab system does not provide error messages on this.

Currently, the Lmod software modules extension might also accidently be used to destroy the functionality of the whole JupyterLab instance.
Unloading the \verb|Jupyter/2019a-Python-3.6.8| modules breaks the functionality of the whole set of default Jupyter kernels.

% THIS PARAGRAPH IS SOMEHOW OUT OF PLACE HERE. I DON'T THINK WE NEED IT AT ALL?  -- WILLI
%
% In the software extension module several Jupyter software environments are available but it remains unclear if stable JupyterLab instance software environments are established:
% %
% \begin{minted}[breaklines,breakanywhere]{bash}
% $ module -t --redirect avail | grep -i jupyter
% Jupyter/
% Jupyter/2019a-rc17-Python-3.6.8
% Jupyter/2019a-rc18-Python-3.6.8
% Jupyter/2019a-rc19-Python-3.6.8
% Jupyter/2019a-rc20-Python-3.6.8
% Jupyter/2019a-rc21-Python-3.6.8
% Jupyter/2019a-rc22-Python-3.6.8
% Jupyter/2019a-rc23-Python-3.6.8
% Jupyter/2019a-rc30-Python-3.6.8
% Jupyter/2019a-rc31-Python-3.6.8
% Jupyter/2019a-devel-Python-3.6.8
% Jupyter/2019a-Python-3.6.8-damian
% Jupyter/2019a-Python-3.6.8
% \end{minted}

For example, unloading the \verb|Jupyter/2019a-Python-3.6.8| module does not unload its dependent modules, even though they are activated by the \verb|load()| command, for which Lmod intended behaviour\footnote{see \href{https://lmod.readthedocs.io/en/latest/098_dependent_modules.html}{Lmod 8.3.1 documentation} on dependencies} is to unload dependencies.
If this a problem on the Lmod side\footnote{on JUWELS Lmod 7.7.38 is installed} or by software module architecture design is not clear at this point.


\subsection{Custom Python kernels}

Modifications to the default JupyterLab software environment of the Python kernel can also be done without loading software environment modules.
The status quo of bringing your own Python packages to the JupyterHub system is evaluated here.
The system designers propose the usage of a \href{https://docs.python.org/3/library/venv.html}{virtual Python environment} as an \href{https://ipython.org/ipython-doc/3/development/kernels.html#kernelspecs}{IPython kernel}.

\subsubsection{Default packages}
\label{sect:python-kernel:default-packages}

The default Python environment as defined by the above software modules contains 314 packages.
The following 154 are currently\footnote{February 19th, 2020} up-to-date:
%
\begin{minted}{bash}
$ pip list --uptodate
Package                            Version
---------------------------------- ---------
aiohttp                            3.6.2
alabaster                          0.7.12
ansiwrap                           0.8.4
apipkg                             1.5
appdirs                            1.4.3
async-generator                    1.10
async-timeout                      3.0.1
atomicwrites                       1.3.0
backcall                           0.1.0
backports.shutil-get-terminal-size 1.0.0
backports.tempfile                 1.0
backports.weakref                  1.0.post1
bleach                             3.1.0
blist                              1.3.6
certipy                            0.1.3
chardet                            3.0.4
Click                              7.0
colorcet                           2.0.2
commonmark                         0.9.1
entrypoints                        0.3
enum34                             1.1.6
fsspec                             0.6.2
funcsigs                           1.0.2
grako                              3.99.9
graphviz                           0.13.2
h5py                               2.10.0
HeapDict                           1.0.1
html5lib                           1.0.1
hyperlink                          19.0.0
idna-ssl                           1.1.0
incremental                        17.5.0
ipydatawidgets                     4.0.1
ipynb                              0.5.1
ipyparallel                        6.2.4
ipython-genutils                   0.2.0
ipythonblocks                      1.9.0
ipyvolume                          0.5.2
ipyvuetify                         1.1.1
ipywebrtc                          0.5.0
ipywidgets                         7.5.1
isodate                            0.6.0
itk-core                           5.0.1
itk-filtering                      5.0.1
itk-io                             5.0.1
itk-meshtopolydata                 0.5.1
itk-numerics                       5.0.1
itk-registration                   5.0.1
itk-segmentation                   5.0.1
itsdangerous                       1.1.0
joblib                             0.14.1
jsfileupload                       0.1.0
jupyter-bokeh                      1.1.1
jupyter-client                     5.3.4
jupyter-contrib-core               0.3.3
jupyter-contrib-nbextensions       0.5.1
jupyter-highlight-selected-word    0.2.0
jupyter-latex-envs                 1.4.6
jupyter-nbextensions-configurator  0.4.1
jupyter-server-proxy               1.2.0
jupyterlab-github                  1.0.0
jupyterlab-gitlab                  0.2.0
jupyterlab-iframe                  0.2.1
jupyterlab-latex                   1.0.0
jupyterlab-launcher                0.13.1
jupyterlab-pygments                0.1.0
jupyterlab-quickopen               0.3.0
jupyterlab-server                  1.0.6
jupyterlab-slurm                   1.0.5
jupyterlmod                        1.7.5
jupyterthemes                      0.20.0
liac-arff                          2.4.0
locket                             0.2.0
lockfile                           0.12.2
MarkupSafe                         1.1.1
mistune                            0.8.4
mpmath                             1.1.0
multipledispatch                   0.6.0
nbconvert                          5.6.1
nbdime                             1.1.0
netaddr                            0.7.19
netCDF4                            1.5.3
netifaces                          0.10.9
nose                               1.3.7
oauthlib                           3.1.0
olefile                            0.46
pamela                             1.0.0
pandocfilters                      1.4.2
partd                              1.1.0
patsy                              0.5.1
paycheck                           1.0.2
pickleshare                        0.7.5
ply                                3.11
prometheus-client                  0.7.1
prov                               1.5.3
ptyprocess                         0.6.0
pvlink                             0.1.2
pyasn1                             0.4.8
pyasn1-modules                     0.2.8
pycodestyle                        2.5.0
pycparser                          2.19
pycrypto                           2.6.1
pyct                               0.4.6
pydot                              1.4.1
pydotplus                          2.0.2
pydub                              0.23.1
pymc3                              3.8
PyNaCl                             1.3.0
pyOpenSSL                          19.1.0
python-editor                      1.0.4
python-gflags                      3.1.2
pythreejs                          2.1.1
PyWavelets                         1.1.1
rdflib                             4.2.2
recommonmark                       0.6.0
retrying                           1.3.3
scandir                            1.10.0
scikit-image                       0.16.2
selenium                           3.141.0
Send2Trash                         1.5.0
service-identity                   18.1.0
sidecar                            0.3.0
simpervisor                        0.3
simplegeneric                      0.8.1
singledispatch                     3.4.0.3
smmap2                             2.0.5
smopy                              0.0.7
snowballstemmer                    2.0.0
sortedcontainers                   2.1.0
sphinx-rtd-theme                   0.4.3
tenacity                           6.0.0
testpath                           0.4.4
textwrap3                          0.9.2
Theano                             1.0.4
toml                               0.10.0
tornado                            6.0.3
traitlets                          4.3.3
traittypes                         0.2.1
ujson                              1.35
vcversioner                        2.16.0.0
vega-datasets                      0.8.0
version-information                1.0.3
vincent                            0.4.4
voila-material                     0.2.5
webencodings                       0.5.1
widgetsnbextension                 3.5.1
wslink                             0.1.11
zict                               1.0.0
\end{minted}
%
The following 160 packages are currently outdated:
%
\begin{minted}{bash}
$ pip list --outdated
Package                   Version   Latest     Type
------------------------- --------- ---------- -----
absl-py                   0.8.1     0.9.0      sdist
alembic                   1.0.8     1.4.0      sdist
altair                    3.3.0     4.0.1      wheel
appmode                   0.6.0     0.7.0      sdist
argcomplete               1.9.5     1.11.1     wheel
arviz                     0.5.1     0.6.1      wheel
asn1crypto                0.24.0    1.3.0      wheel
attrs                     19.1.0    19.3.0     wheel
autobahn                  19.10.1   20.2.1     wheel
Automat                   0.8.0     20.2.0     wheel
autopep8                  1.4.4     1.5        sdist
Babel                     2.7.0     2.8.0      wheel
bash-kernel               0.7.1     0.7.2      wheel
bcrypt                    3.1.6     3.1.7      wheel
bitstring                 3.1.5     3.1.6      wheel
bokeh                     1.3.4     1.4.0      sdist
bqplot                    0.11.9    0.12.3     wheel
branca                    0.3.1     0.4.0      wheel
certifi                   2019.3.9  2019.11.28 wheel
cffi                      1.12.2    1.14.0     wheel
cftime                    1.0.4.2   1.1.0      wheel
cloudpickle               0.8.1     1.3.0      wheel
ClusterShell              1.8       1.8.3      wheel
colorama                  0.4.1     0.4.3      wheel
configparser              3.7.4     4.0.2      wheel
Cython                    0.29.6    0.29.15    wheel
dask                      2.6.0     2.10.1     wheel
dask-labextension         1.0.3     1.1.0      sdist
datashader                0.9.0     0.10.0     wheel
deap                      1.2.2     1.3.1      wheel
decorator                 4.4.0     4.4.1      wheel
dicom-upload              0.1.1     0.1.2      wheel
distributed               2.6.0     2.10.0     wheel
docutils                  0.15.2    0.16       wheel
ecdsa                     0.13      0.15       wheel
execnet                   1.6.0     1.7.1      wheel
Flask                     1.0.2     1.1.1      wheel
future                    0.18.1    0.18.2     sdist
gitdb2                    2.0.6     3.0.2      wheel
GitPython                 3.0.4     3.0.8      wheel
holoviews                 1.12.6    1.12.7     wheel
idna                      2.8       2.9        wheel
imageio                   2.6.1     2.8.0      wheel
imagesize                 1.1.0     1.2.0      wheel
importlib-metadata        0.23      1.5.0      wheel
ipaddress                 1.0.22    1.0.23     wheel
ipykernel                 5.1.3     5.1.4      wheel
ipyleaflet                0.11.4    0.12.2     wheel
ipympl                    0.3.3     0.4.1      wheel
ipython                   7.9.0     7.12.0     wheel
ipyvue                    1.0.0     1.1.0      sdist
itkwidgets                0.22.0    0.25.3     wheel
jedi                      0.13.3    0.16.0     wheel
Jinja2                    2.10      2.11.1     wheel
json5                     0.8.5     0.9.1      wheel
jsonschema                3.1.1     3.2.0      wheel
julia                     0.5.0     0.5.1      wheel
jupyter-core              4.6.1     4.6.3      wheel
jupyter-server            0.1.1     0.2.1      wheel
jupyterhub                1.0.0     1.1.0      wheel
jupyterlab                1.2.1     1.2.6      wheel
jupyterlab-code-formatter 0.6.1     1.1.0      sdist
jupyterlab-git            0.8.1     0.9.0      wheel
kiwisolver                1.0.1     1.1.0      wheel
lesscpy                   0.13.0    0.14.0     wheel
line-profiler             2.1.2     3.0.2      wheel
llvmlite                  0.30.0    0.31.0     wheel
lxml                      4.3.3     4.5.0      wheel
Mako                      1.0.8     1.1.1      sdist
matplotlib                3.0.3     3.1.3      wheel
memory-profiler           0.55.0    0.57.0     sdist
mock                      2.0.0     4.0.1      wheel
more-itertools            7.0.0     8.2.0      wheel
msgpack                   0.6.2     1.0.0      wheel
multidict                 4.5.2     4.7.4      wheel
nbformat                  4.4.0     5.0.4      wheel
nbresuse                  0.3.2     0.3.3      wheel
networkx                  2.3       2.4        wheel
notebook                  6.0.2     6.0.3      wheel
numba                     0.46.0    0.48.0     wheel
numexpr                   2.7.0     2.7.1      wheel
numpy                     1.15.2    1.18.1     wheel
packaging                 19.0      20.1       wheel
pandas                    0.24.2    1.0.1      wheel
papermill                 1.2.1     2.0.0      wheel
param                     1.9.2     1.9.3      wheel
paramiko                  2.4.2     2.7.1      wheel
parso                     0.3.4     0.6.1      wheel
pathlib2                  2.3.3     2.3.5      wheel
pathspec                  0.6.0     0.7.0      wheel
pbr                       5.1.3     5.4.4      wheel
pexpect                   4.7.0     4.8.0      wheel
Pillow                    6.0.0     7.0.0      wheel
pip                       19.0.3    20.0.2     wheel
plotly                    4.2.1     4.5.0      wheel
pluggy                    0.9.0     0.13.1     wheel
prompt-toolkit            2.0.9     3.0.3      wheel
protobuf                  3.7.1     3.11.3     wheel
psutil                    5.6.1     5.7.0      sdist
psycopg2                  2.7.7     2.8.4      sdist
py                        1.8.0     1.8.1      wheel
pydicom                   1.3.0     1.4.1      wheel
Pygments                  2.4.2     2.5.2      wheel
PyHamcrest                1.9.0     2.0.0      wheel
pyparsing                 2.3.1     2.4.6      wheel
pyrsistent                0.14.11   0.15.7     sdist
pytest                    4.4.0     5.3.5      wheel
pytest-forked             1.0.2     1.1.3      wheel
pytest-runner             4.4       5.2        wheel
pytest-xdist              1.27.0    1.31.0     wheel
python-dateutil           2.8.0     2.8.1      wheel
python-dotenv             0.10.3    0.11.0     wheel
python-oauth2             1.1.0     1.1.1      sdist
pytz                      2018.9    2019.3     wheel
pyviz-comms               0.7.2     0.7.3      wheel
PyYAML                    5.1.2     5.3        sdist
pyzmq                     18.1.0    18.1.1     wheel
regex                     2019.11.1 2020.2.18  wheel
requests                  2.21.0    2.22.0     wheel
rise                      5.5.1     5.6.0      wheel
Rtree                     0.8.3     0.9.4      sdist
scikit-learn              0.22      0.22.1     wheel
scipy                     1.2.1     1.4.1      wheel
seaborn                   0.9.0     0.10.0     wheel
setuptools                45.0.0    45.2.0     wheel
setuptools-scm            3.2.0     3.5.0      wheel
simplejson                3.16.0    3.17.0     sdist
six                       1.12.0    1.14.0     wheel
SPARQLWrapper             1.8.2     1.8.5      wheel
Sphinx                    1.8.5     2.4.2      wheel
sphinxcontrib-websupport  1.1.2     1.2.0      wheel
SQLAlchemy                1.3.1     1.3.13     sdist
statsmodels               0.10.2    0.11.0     wheel
sympy                     1.3       1.5.1      wheel
TatSu                     4.2.6     4.4.0      wheel
tblib                     1.5.0     1.6.0      wheel
terminado                 0.8.2     0.8.3      wheel
tikzplotlib               0.8.4     0.9.1      wheel
toolz                     0.9.0     0.10.0     sdist
tqdm                      4.41.0    4.43.0     wheel
traits                    5.0.0     6.0.0      sdist
Twisted                   19.7.0    19.10.0    wheel
txaio                     18.8.1    20.1.1     wheel
typed-ast                 1.4.0     1.4.1      wheel
typing                    3.6.6     3.7.4.1    wheel
typing-extensions         3.7.4     3.7.4.1    wheel
urllib3                   1.24.1    1.25.8     wheel
voila                     0.1.14    0.1.20     wheel
voila-gridstack           0.0.6     0.0.8      wheel
voila-vuetify             0.1.1     0.2.2      sdist
wcwidth                   0.1.7     0.1.8      wheel
Werkzeug                  0.15.1    1.0.0      wheel
wheel                     0.33.1    0.34.2     wheel
xarray                    0.12.1    0.15.0     wheel
XlsxWriter                1.1.5     1.2.7      wheel
yapf                      0.28.0    0.29.0     wheel
yarl                      1.3.0     1.4.2      wheel
zipp                      0.6.0     3.0.0      wheel
zope.interface            4.6.0     4.7.1      wheel
zstandard                 0.12.0    0.13.0     wheel
\end{minted}

NOTE THAT THERE'S REDUNDANT STUFF IN THE ENV AS WELL. FOR EXAMPLE, ENUM34 IS A BACKPORT OF PY34 FUNCTIONALITY TO 33 AND PY2.

While slightly outdated Python packages might not be problematic for many use cases, other use cases might be strongly dependent on recent developments in the quickly evolving scientific Python ecosystem.

Furthermore, there are use cases that require running Jupyter Notebook code in a version specific package environment.
For example, because newer package versions break older package syntax or functionality, or because of the need to exactly repeat a former analysis.

For these reasons the possibility of setting up user specific and standalone package environments is a fundamental requirement for an all purpose JupyterHub/Lab system.

\subsubsection{Python virtual environment}
\label{sect:python-kernel:virtual-environment}

The JupyterHub login page provides a link to a \href{https://jupyter-jsc.fz-juelich.de/hub/static/files/kernel.html}{\emph{Setup your own kernel}} document where instructions on how to define virtual Python software environments and registering these as IPython/Jupyter kernels are given.

The document proposes to create the Python virtual environment inside the activated default software environment described in section \ref{sect:system-description} using

\begin{minted}{text}
$ pwd
/p/project/cesmtst/hoeflich1/kernels
$ python3 -m venv jupyterhub-evaluation-v2020.02.24.1
$ source jupyterhub-evaluation-v2020.02.24.1/bin/activate
\end{minted}

The pip package manager that configures the virtual Python environment looks up installed packages in the Python path
%
\begin{minted}[breaklines,breakanywhere]{text}
(jupyterhub-evaluation-v2020.02.24.1) $ echo $PYTHONPATH
/gpfs/software/juwels/stages/Devel-2019a/software/Jupyter/2019a-gcccoremkl-8.3.0-2019.3.199-Python-3.6.8/lib/python3.6/site-packages:/gpfs/software/juwels/stages/Devel-2019a/software/netcdf4-python/1.5.3-gcccoremkl-8.3.0-2019.3.199-serial-Python-3.6.8/lib/python3.6/site-packages:/gpfs/software/juwels/stages/Devel-2019a/software/ITK/5.0.1-gcccoremkl-8.3.0-2019.3.199-Python-3.6.8/lib/python3.6/site-packages:/gpfs/software/juwels/stages/Devel-2019a/software/VTK/8.2.0-gcccoremkl-8.3.0-2019.3.199-Python-3.6.8/lib64/python3.6/site-packages:/gpfs/software/juwels/stages/Devel-2019a/software/SciPy-Stack/2019a-gcccoremkl-8.3.0-2019.3.199-Python-3.6.8/lib/python3.6/site-packages:/gpfs/software/juwels/stages/Devel-2019a/software/Python/3.6.8-GCCcore-8.3.0/easybuild/python:/gpfs/software/juwels/stages/Devel-2019a/software/Python/3.6.8-GCCcore-8.3.0/lib/python3.6/site-packages
\end{minted}
%
and the documentation proposes to prepend the file system location of the user specific virtual Python environment using
%
\begin{minted}[breaklines,breakanywhere]{text}
(jupyterhub-evaluation-v2020.02.24.1) $ export PYTHONPATH=/p/project/cesmtst/hoeflich1/kernels/jupyterhub-evaluation-v2020.02.24.1/lib/python3.6/site-packages:${PYTHONPATH}
\end{minted}
%
effectively allowing the user to overload Python packages inside the existing default Python environment described in section \ref{sect:python-kernel:default-packages}.

Before installing any desired packages, the installation of the IPython kernel package using the \verb|--ignore-installed| pip flag is suggested
%
\begin{minted}[breaklines,breakanywhere]{text}
(jupyterhub-evaluation-v2020.02.24.1) $ pip install --ignore-installed ipykernel
Collecting ipykernel
  Using cached https://files.pythonhosted.org/packages/d7/62/d1a5d654b7a21bd3eb99be1b59a608cc18a7a08ed88495457a87c40a0495/ipykernel-5.1.4-py3-none-any.whl
Collecting traitlets>=4.1.0 (from ipykernel)
  Using cached https://files.pythonhosted.org/packages/ca/ab/872a23e29cec3cf2594af7e857f18b687ad21039c1f9b922fac5b9b142d5/traitlets-4.3.3-py2.py3-none-any.whl
Collecting ipython>=5.0.0 (from ipykernel)
  Using cached https://files.pythonhosted.org/packages/b8/6d/1e3e335e767fc15a2047a008e27df31aa8bcf11c6f3805d03abefc69aa88/ipython-7.12.0-py3-none-any.whl
Collecting tornado>=4.2 (from ipykernel)
Collecting jupyter-client (from ipykernel)
  Downloading https://files.pythonhosted.org/packages/40/75/4c4eb43749e59db3c1c7932b50eaf8c4b8219b1b5644fe379ea796f8dbe5/jupyter_client-6.0.0-py3-none-any.whl (104kB)
    100% |||||||||||||||||||||||||||||||||| 112kB 4.0MB/s
Collecting decorator (from traitlets>=4.1.0->ipykernel)
  Using cached https://files.pythonhosted.org/packages/8f/b7/f329cfdc75f3d28d12c65980e4469e2fa373f1953f5df6e370e84ea2e875/decorator-4.4.1-py2.py3-none-any.whl
Collecting ipython-genutils (from traitlets>=4.1.0->ipykernel)
  Using cached https://files.pythonhosted.org/packages/fa/bc/9bd3b5c2b4774d5f33b2d544f1460be9df7df2fe42f352135381c347c69a/ipython_genutils-0.2.0-py2.py3-none-any.whl
Collecting six (from traitlets>=4.1.0->ipykernel)
  Using cached https://files.pythonhosted.org/packages/65/eb/1f97cb97bfc2390a276969c6fae16075da282f5058082d4cb10c6c5c1dba/six-1.14.0-py2.py3-none-any.whl
Collecting backcall (from ipython>=5.0.0->ipykernel)
Collecting setuptools>=18.5 (from ipython>=5.0.0->ipykernel)
  Using cached https://files.pythonhosted.org/packages/3d/72/1c1498c1e908e0562b1e1cd30012580baa7d33b5b0ffdbeb5fde2462cc71/setuptools-45.2.0-py3-none-any.whl
Collecting pickleshare (from ipython>=5.0.0->ipykernel)
  Using cached https://files.pythonhosted.org/packages/9a/41/220f49aaea88bc6fa6cba8d05ecf24676326156c23b991e80b3f2fc24c77/pickleshare-0.7.5-py2.py3-none-any.whl
Collecting jedi>=0.10 (from ipython>=5.0.0->ipykernel)
  Using cached https://files.pythonhosted.org/packages/01/67/333e2196b70840f411fd819407b4e98aa3150c2bd24c52154a451f912ef2/jedi-0.16.0-py2.py3-none-any.whl
Collecting prompt-toolkit!=3.0.0,!=3.0.1,<3.1.0,>=2.0.0 (from ipython>=5.0.0->ipykernel)
  Using cached https://files.pythonhosted.org/packages/f5/22/f00412fafc68169054cc623a35c32773f22b403ddbe516c8adfdecf25341/prompt_toolkit-3.0.3-py3-none-any.whl
Collecting pygments (from ipython>=5.0.0->ipykernel)
  Using cached https://files.pythonhosted.org/packages/be/39/32da3184734730c0e4d3fa3b2b5872104668ad6dc1b5a73d8e477e5fe967/Pygments-2.5.2-py2.py3-none-any.whl
Collecting pexpect; sys_platform != "win32" (from ipython>=5.0.0->ipykernel)
  Using cached https://files.pythonhosted.org/packages/39/7b/88dbb785881c28a102619d46423cb853b46dbccc70d3ac362d99773a78ce/pexpect-4.8.0-py2.py3-none-any.whl
Collecting python-dateutil>=2.1 (from jupyter-client->ipykernel)
  Using cached https://files.pythonhosted.org/packages/d4/70/d60450c3dd48ef87586924207ae8907090de0b306af2bce5d134d78615cb/python_dateutil-2.8.1-py2.py3-none-any.whl
Collecting jupyter-core>=4.6.0 (from jupyter-client->ipykernel)
  Using cached https://files.pythonhosted.org/packages/63/0d/df2d17cdf389cea83e2efa9a4d32f7d527ba78667e0153a8e676e957b2f7/jupyter_core-4.6.3-py2.py3-none-any.whl
Collecting pyzmq>=13 (from jupyter-client->ipykernel)
  Using cached https://files.pythonhosted.org/packages/94/07/cee3d328a2e13f9de1c2b62cced7a389b61ac81424f2e377f3dc9d668282/pyzmq-18.1.1-cp36-cp36m-manylinux1_x86_64.whl
Collecting parso>=0.5.2 (from jedi>=0.10->ipython>=5.0.0->ipykernel)
  Using cached https://files.pythonhosted.org/packages/ec/bb/3b6c9f604ac40e2a7833bc767bd084035f12febcbd2b62204c5bc30edf97/parso-0.6.1-py2.py3-none-any.whl
Collecting wcwidth (from prompt-toolkit!=3.0.0,!=3.0.1,<3.1.0,>=2.0.0->ipython>=5.0.0->ipykernel)
  Using cached https://files.pythonhosted.org/packages/58/b4/4850a0ccc6f567cc0ebe7060d20ffd4258b8210efadc259da62dc6ed9c65/wcwidth-0.1.8-py2.py3-none-any.whl
Collecting ptyprocess>=0.5 (from pexpect; sys_platform != "win32"->ipython>=5.0.0->ipykernel)
  Using cached https://files.pythonhosted.org/packages/d1/29/605c2cc68a9992d18dada28206eeada56ea4bd07a239669da41674648b6f/ptyprocess-0.6.0-py2.py3-none-any.whl
Installing collected packages: decorator, ipython-genutils, six, traitlets, backcall, setuptools, pickleshare, parso, jedi, wcwidth, prompt-toolkit, pygments, ptyprocess, pexpect, ipython, tornado, python-dateutil, jupyter-core, pyzmq, jupyter-client, ipykernel
Successfully installed backcall-0.1.0 decorator-4.4.1 ipykernel-5.1.4 ipython-7.12.0 ipython-genutils-0.2.0 jedi-0.16.0 jupyter-client-6.0.0 jupyter-core-4.6.3 parso-0.6.1 pexpect-4.8.0 pickleshare-0.7.5 prompt-toolkit-3.0.3 ptyprocess-0.6.0 pygments-2.5.2 python-dateutil-2.8.1 pyzmq-18.1.1 setuptools-45.2.0 six-1.14.0 tornado-6.0.3 traitlets-4.3.3 wcwidth-0.1.8
You are using pip version 18.1, however version 20.0.2 is available.
You should consider upgrading via the 'pip install --upgrade pip' command.
\end{minted}
%
which installs the IPython kernel package and all dependencies into the virtual Python environment either by downloading an up-to-date version or by copying from the default environment packages.

For evaluation purposes we proceed with the example task of upgrading the outdated (for the package versions see section \ref{sect:python-kernel:default-packages}) Dask package
%
\begin{minted}[breaklines,breakanywhere]{text}
(jupyterhub-evaluation-v2020.02.24.1) $ pip install --upgrade dask
Collecting dask
  Downloading https://files.pythonhosted.org/packages/c2/7a/22ffaff40b7c912cac6956e18d38c686b6b5756179e9c09e8e6bf7810aad/dask-2.11.0-py3-none-any.whl (785kB)
    100% |||||||||||||||||||||||||||||||||| 788kB 6.2MB/s
Installing collected packages: dask
  Found existing installation: dask 2.6.0
    Not uninstalling dask at /gpfs/software/juwels/stages/Devel-2019a/software/Jupyter/2019a-gcccoremkl-8.3.0-2019.3.199-Python-3.6.8/lib/python3.6/site-packages/dask-2.6.0-py3.6.egg, outside environment /p/project/cesmtst/hoeflich1/kernels/jupyterhub-evaluation-v2020.02.24.1
    Can't uninstall 'dask'. No files were found to uninstall.
Successfully installed dask-2.11.0
You are using pip version 18.1, however version 20.0.2 is available.
You should consider upgrading via the 'pip install --upgrade pip' command.
\end{minted}

To register the above virtual Python environment as Jupyter kernel the documentation suggests using the following IPython kernel command
%
\begin{minted}[breaklines,breakanywhere]{text}
(jupyterhub-evaluation-v2020.02.24.1) $ python3 -m ipykernel install --user --name=jupyterhub-evaluation-v2020.02.24.1
\end{minted}
%
which installs a default Jupyter kernel configuration into the user directory:
%
\begin{minted}[breaklines,breakanywhere]{text}
$ pwd
/p/home/jusers/hoeflich1/juwels/.local/share/jupyter/kernels/jupyterhub-evaluation-v2020.02.24.1
$ cat kernel.json
{
 "argv": [
  "/p/project/cesmtst/hoeflich1/kernels/jupyterhub-evaluation-v2020.02.24.1/bin/python3",
  "-m",
  "ipykernel_launcher",
  "-f",
  "{connection_file}"
 ],
 "display_name": "jupyterhub-evaluation-v2020.02.24.1",
 "language": "python"
}
\end{minted}

This already defines a working IPython kernel for JupyterLab instances, however, only the default Python environment package collection described in section \ref{sect:python-kernel:default-packages} is available.

To use the virtual Python environment that was created above an executable kernel launch script needs to be set up
%
\begin{minted}[breaklines,breakanywhere]{text}
$ pwd
/p/project/cesmtst/hoeflich1/kernels/jupyterhub-evaluation-v2020.02.24.1
$ cat kernel.sh
#!/bin/bash
module purge --force
module use $OTHERSTAGES
module load Stages/Devel-2019a GCCcore/.8.3.0
module load Jupyter/2019a-Python-3.6.8
source /p/project/cesmtst/hoeflich1/kernels/jupyterhub-evaluation-v2020.02.24.1/bin/activate
export PYTHONPATH=/p/project/cesmtst/hoeflich1/kernels/jupyterhub-evaluation-v2020.02.24.1/lib/python3.6/site-packages:${PYTHONPATH}
exec python -m ipykernel $@
$ chmod +x kernel.sh
\end{minted}
%
and the IPython kernel is launched after both the default and virtual Python environments are activated.
The kernel configuration file needs to point to the kernel launch script
%
\begin{minted}[breaklines,breakanywhere]{text}
$ pwd
/p/home/jusers/hoeflich1/juwels/.local/share/jupyter/kernels/jupyterhub-evaluation-v2020.02.24.1
$ cat kernel.sh
{
 "argv": [
  "/p/project/cesmtst/hoeflich1/kernels/jupyterhub-evaluation-v2020.02.24.1/kernel.sh",
  "-f",
  "{connection_file}"
 ],
 "display_name": "jupyterhub-evaluation-v2020.02.24.1",
 "language": "python"
}
\end{minted}
%
instead of directly calling the IPython kernel.
The Python packages from the default environment extended by packages from the virtual environment are then available inside Jupyter notebooks.

\subsubsection{Detailed remarks}

\begin{itemize}

  \item not clear why the installation of the IPython kernel package into the virtual environment is suggested;
  this is not strictly necessary because of how the Python package path environment variable is set;
  Python packages of the virtual environment only overload/extend existing default packages and the IPython kernel package is already available in the default Python environment

  \item the \verb|--ignore-installed| pip flag is not a recommended feature\footnote{https://stackoverflow.com/questions/51913361/difference-between-pip-install-options-ignore-installed-and-force-reinstall};
  already existing packages in the user file system location are not properly uninstalled, which could lead to the accumulation of orphaned files that would need to be cleaned up manually

  \item the \verb|--ignore-installed| flag has an unnecessarily negative impact upon user file system inode counts;
  all packages and their dependencies are copied into the virtual Python environment even though exactly needed versions are already available in the default environment

  \item to keep the virtual environment small, for packages that are already installed it would be better to use the \verb|--upgrade| flag or to define a specific package version

  \item the 314 package default Python environment is currently always carried along;
  Python virtual environments are a Python base feature and only need a working core Python interpreter/environment and pip and setuptools packages installed;
  it would thus be possible to define very light-weight user specific virtual Python environments as kernels

\end{itemize}

\subsection{Jupyter kernel suggestions}
\label{sect:jupyter-kernel-suggestions}

Overloading single packages in the 314 packages default Python environment might not fulfil every user's software requirements (see section \ref{sect:python-kernel:default-packages} for reasons).
Therefore, a scripted light-weight virtual Python environment solution is described in the following.
As virtual Python environments are not purely standalone, we further explore the possibility of using a conda managed and a container-based environment as Jupyter kernels.

\subsubsection{Light-weight virtual Python environment}
\label{sect:suggested-python-virtual-environment}

The kernel launcher script approach from section \ref{sect:python-kernel:virtual-environment} is used.
As base environment only the more light-weight Python software environment module is loaded.
The following script is designed to install, register and execute a virtual Python environment as Jupyter kernel:
%
\begin{minted}[breaklines,breakanywhere]{bash}
#!/bin/bash

kernel_name=dask-jobqueue-v2020.02.24.3
python_packages='dask[complete] dask-jobqueue'

jutil env activate -p cesmtst
base_directory=$PROJECT/$USER

module purge
module load GCC/8.3.0
module load Python/3.6.8

if [ "${1}" == "install" ]; then python3 -m venv ${base_directory}/kernels/${kernel_name}; fi

source ${base_directory}/kernels/${kernel_name}/bin/activate
export PYTHONPATH=${base_directory}/kernels/${kernel_name}/lib/python3.6/site-packages

if [ "${1}" == "install" ]; then
  # Upgrade pip and install Python packages
  pip install --upgrade pip
  pip install ipykernel ${python_packages}
  # Register this virtual Python environment as Jupyter kernel
  python3 -m ipykernel install --user --name=${kernel_name}
  # Establish standalone-ness of this virtual Python environment
  cp kernel.sh ${base_directory}/kernels/${kernel_name}/
  jupyter_kernelspec_path=${HOME}/.local/share/jupyter/kernels
  this_kernel=${jupyter_kernelspec_path}/${kernel_name,,}/kernel.json # lower case conversion
  sed -i "s|/bin/python3|/kernel.sh|g" ${this_kernel}
  sed -i "/-m/d" ${this_kernel}
  sed -i "/ipykernel_launcher/d" ${this_kernel}
else
  exec python -m ipykernel $@
fi

\end{minted}

%https://stackoverflow.com/questions/1382925/virtualenv-no-site-packages-and-pip-still-finding-global-packages
%https://docs.python.org/3/tutorial/venv.html
%https://packaging.python.org/guides/installing-using-pip-and-virtual-environments/

%https://stackoverflow.com/questions/52060533/does-pip-install-add-current-directory-to-pythonpath
%https://www.dabapps.com/blog/introduction-to-pip-and-virtualenv-python/
%https://able.bio/rhett/python-virtual-environments-with-virtualenv--63mmpmp

\subsubsection{Conda managed environment}
\label{sect:conda-environment}

The downside of Python virtual environments is that they depend upon the parent Python interpreter features only.
There might, however, be use cases that require not only custom Python packages, but also custom Python interpreter versions.
The \href{https://conda.io/en/latest/}{Conda package manager} is one way to solve this.

To define a Conda environment as kernelm, the launcher script approach described in \ref{sect:python-kernel:virtual-environment} is used.
The script essentially activates a Conda Python environment that has the IPython kernel package installed and could be placed into the default Jupyter kernel location:
%
\begin{minted}[breaklines,breakanywhere]{text}
$ pwd
/p/home/jusers/hoeflich1/juwels/.local/share/jupyter/kernels/miniconda3-kernel
$ cat kernel.sh
#!/bin/bash
module purge --force
source /p/project/cesmtst/hoeflich1/miniconda3/bin/activate Dask-jobqueue_v2020.02.10
exec python -m ipykernel $@
\end{minted}

After setting up a kernel configuration file manually
%
\begin{minted}[breaklines,breakanywhere]{text}
$ cat kernel.json
{
 "argv": [
  "/p/home/jusers/hoeflich1/juwels/.local/share/jupyter/kernels/miniconda3-kernel/kernel.sh",
  "-f",
  "{connection_file}"
 ],
 "display_name": "miniconda3-kernel",
 "language": "python"
}
\end{minted}
%
the conda-maintained environment works as Jupyter kernel.

%https://github.com/jupyterhub/jupyterhub/issues/847
%https://jupyter-client.readthedocs.io/en/stable/kernels.html
%https://github.com/pyenv/pyenv

\subsubsection{Singularity environment}
\label{sect:container-based-environment}

Containers provide a flexible way of deploying isolated software environments, thus enabling standalone analysis and compute environments and scientific reproducibility.
They can also be used as Jupyter kernels.

(How exactly this works should be described here. Further information and starting points are found in \href{https://github.com/ExaESM-WP4/JupyterHub-Evaluation-Whitepaper/issues/6}{this Github issue}.)

\subsection{Summary}
\label{sect:working-envs:summary}

Here we summarize the status quo of working environment selection, modification and specification for JUWELS JupyterLab instances.
Specific proposals to achieve better robustness and flexibility, compatible with the chosen system software architecture, are also given.

Principal other user working environment managing approaches are discussed at the end of the report. OR SHOULD WE NOT DO THIS? (JupyterLab instances could also be started inside containers. Currently, this approach is used only on the HDF cloud. Should this be made possible on JURECA and JUWELS, too? What is pro and contra here? From a user and system administrator perspective?)

\subsubsection{Selection, modification and specification}

Working environment selection is established based on Jupyter kernels, available per default are currently a Bash, Javascript, Julia and Python kernel, and several C++ kernels.
The default Python package environment can be extended and modified using the Lmod software environment Jupyter extension available in the JupyterLab sidebar, however, most of the available Python modules are not compatible with the default environment, quickly leaving the user with broken Python kernel functionality.
THERE SHOULD BE MORE DETAILS FROM ABOVE INVESTIGATIONS HERE.
The central problem is the entirely missing documentation on what can or should be achieved via the software environment module extension.

Modification of the default Python environment can also be done by setting up a virtual Python environment, a documentation is provided that explains registering it as further Jupyter kernel.
However, the approach that is described only allows the user to modify certain Python packages inside the extensive default Python environment.
Specifying a more standalone and light-weight virtual Python environment as Jupyter kernel was shown to be possible here, but is currently not documented.

The specification of completely standalone working environments has been explored in sections \ref{sect:conda-environment} and \ref{sect:container-based-environment}.
Both a Conda package manager and a container based Jupyter kernel approach are presented.
They extend the documented approaches by the possibility of deploying isolated software environments, enabling convenient porting of analysis environments and facilitating scientific reproducibility.

\subsubsection{Suggestions}

\begin{itemize}

  \item set up a software modules hierarchy that is scoped only for the JUWELS JupyterLab functionality;
  provided Lmod software environment modules should be completely exchangeable;
  the user should not be able or should clearly be instructed how not to destroy their Jupyter kernels and JupyterLab instances

  \item on system start-up the \verb|Jupyter/2019a-Python-3.6.8| module automatically establishes the software environment for the complete set of available Jupyter kernels;
  in terms of preventing software environment conflicts, it might make sense to provide only a very minimalistic Jupyter environment;
  the user should not be able to unload the modules that provide basic system functionality;
  a documentation could be displayed on how to activate software environment modules that provide Jupyter kernel functionality;
  thus, users are taught to consciously handle their software environments;
  however, there might be work on the Jupyter kernel launcher tab necessary;
  currently, the launcher tab only shows the Jupyter kernels available on JupyterLab instance start-up;
  kernels that become available after JupyterLab has started can be chosen only from the kernel drop down of open Jupyter notebooks

  \item another approach to robust Jupyter kernels would be to make software module based kernel launcher scripts the default; these would show up in the JupyterLab launcher tab on system start-up; loading and unloading of software environment modules would not have an effect on the kernel software environments; customization of kernel software environments would be entirely by setting up own kernel launcher scripts; in this case good documentation needs to be provided by the system designers/administators; the Lmod software modules extension in the JupyterLab sidebar would be obsolete in this case

  \item provide only a minimalistic Python interpreter software environment module for users' to build their own virtual Python environments and Jupyter kernels;
  this would combine the advantage of using a JUWELS specific Python interpreter (with its potentially higher code execution speeds, see \href{https://www.fz-juelich.de/SharedDocs/Downloads/IAS/JSC/EN/slides/supercomputer-ressources-2019-11/15a-tuning_intel.html}{this example investigation by Intel}) with robust standalone pip package dependency managing

  \item extend the documentation for working environment specification based on the alternatives presented in section \ref{sect:jupyter-kernel-suggestions}

\end{itemize}


\section{Technical details of the DKRZ Jupyter solutions}
\label{app:dkrz-jupyter}

\subsection{HPC system infrastructure details}
\label{app:dkrz-infrastructure}

\begin{minted}[breaklines,breakanywhere]{shell-session}
$ sinfo --Node --long | grep m -c
3405
$ sinfo --Node --long | grep ' 48 ' -c
1604
$ sinfo --Node --long | grep ' 72 ' -c
1801
\end{minted}

\begin{minted}[breaklines,breakanywhere]{shell-session}
$ sinfo --Node --long | grep ' compute ' -c
1513
$ sinfo --Node --long | grep ' compute2 ' -c
1762
$ sinfo --Node --long | grep ' prepost ' -c
43
$ sinfo --Node --long | grep ' shared ' -c
36
$ sinfo --Node --long | grep ' gpu ' -c
21
$ sinfo --Node --long | grep ' miklip ' -c
30
\end{minted}

\subsection{Jupyter working environment}
\label{app:dkrz-jupyter-env}

The following shell sessions happened in a terminal on the Jupyter notebook web interface.
First, the available JupyterHub and Jupyter notebook environment, and Jupter notebook extensions are briefly investigated.

\begin{minted}[breaklines,breakanywhere]{shell-session}
$ jupyterhub --version
0.7.2
$ jupyter-notebook --version
5.1.0rc2
$ jupyter --paths
config:
    /mnt/lustre01/pf/b/b350097/.jupyter
    /sw/rhel6-x64/jupyterhub/python-3.5.4-gcc49/etc/jupyter
    /usr/local/etc/jupyter
    /etc/jupyter
data:
    /sw/rhel6-x64/jupyterhub/jupyter_kernels
    /mnt/lustre01/pf/b/b350097/.local/share/jupyter
    /sw/rhel6-x64/jupyterhub/python-3.5.4-gcc49/share/jupyter
    /usr/local/share/jupyter
    /usr/share/jupyter
runtime:
    /mnt/lustre01/pf/b/b350097/.local/share/jupyter/runtime
\end{minted}

\begin{minted}[breaklines,breakanywhere]{shell-session}
$ jupyter kernelspec list
Available kernels:
  python3               /sw/rhel6-x64/jupyterhub/python-3.5.4-gcc49/lib/python3.5/site-packages/ipykernel/resources
  anaconda2_bleeding    /sw/rhel6-x64/jupyterhub/jupyter_kernels/kernels/anaconda2_bleeding
  anaconda3_bleeding    /sw/rhel6-x64/jupyterhub/jupyter_kernels/kernels/anaconda3_bleeding
  python_2.7.12         /sw/rhel6-x64/jupyterhub/jupyter_kernels/kernels/python_2.7.12
  python_3.5.2          /sw/rhel6-x64/jupyterhub/jupyter_kernels/kernels/python_3.5.2
$ jupyter nbextension list
Known nbextensions:
\end{minted}

Now, the management of the Jupyter software and Jupyter kernel environment on the Mistral HPC system is investigated in more detail.
The Jupyter notebook environment called by the DKRZ JupyterHub is very minimalistic.

\begin{minted}[breaklines,breakanywhere]{shell-session}
$ module list
Currently Loaded Modulefiles:
  1) texlive/2016      2) jupyterhub/.0.1
$ module display jupyterhub/.0.1
-------------------------------------------------------------------
/sw/rhel6-x64/Modules/jupyterhub/.0.1:

module-whatis    jupyterhub
conflict         jupyterhub
prepend-path     PATH /sw/rhel6-x64/jupyterhub/python-3.5.4-gcc49/bin
setenv           JUPYTER_PATH /sw/rhel6-x64/jupyterhub/jupyter_kernels
-------------------------------------------------------------------
$ ls /sw/rhel6-x64/jupyterhub/python-3.5.4-gcc49/bin -lha
total 62K
drwxrwsr-x 2 cvmfs cvmfs 4.0K Jun 19  2018 .
drwxrwsr-x 6 cvmfs cvmfs 4.0K Sep  1  2017 ..
lrwxrwxrwx 1 cvmfs cvmfs    8 Sep  1  2017 2to3 -> 2to3-3.5
-rwxrwxr-x 1 cvmfs cvmfs  134 Sep  1  2017 2to3-3.5
-rwxr-xr-x 1 cvmfs cvmfs  423 Sep  1  2017 alembic
-rwxr-xr-x 1 cvmfs cvmfs  266 Sep  1  2017 chardetect
-rwxrwxr-x 1 cvmfs cvmfs  275 Sep  1  2017 easy_install-3.5
lrwxrwxrwx 1 cvmfs cvmfs    7 Sep  1  2017 idle3 -> idle3.5
-rwxrwxr-x 1 cvmfs cvmfs  132 Sep  1  2017 idle3.5
-rwxr-xr-x 1 cvmfs cvmfs  276 Sep  1  2017 iptest
-rwxr-xr-x 1 cvmfs cvmfs  276 Sep  1  2017 iptest3
-rwxr-xr-x 1 cvmfs cvmfs  269 Sep  1  2017 ipython
-rwxr-xr-x 1 cvmfs cvmfs  269 Sep  1  2017 ipython3
-rwxr-xr-x 1 cvmfs cvmfs  258 Sep  1  2017 jsonschema
-rwxr-xr-x 1 cvmfs cvmfs  264 Sep  1  2017 jupyter
-rwxr-xr-x 1 cvmfs cvmfs  278 Sep  1  2017 jupyter-bundlerextension
-rwxr-xr-x 1 cvmfs cvmfs  100 Sep  1  2017 jupyterhub
-rwxr-xr-x 1 cvmfs cvmfs  139 Sep  1  2017 jupyterhub-singleuser
-rwxr-xr-x 1 cvmfs cvmfs  306 Sep  1  2017 jupyter-kernelspec
-rwxr-xr-x 1 cvmfs cvmfs  264 Sep  1  2017 jupyter-migrate
-rwxr-xr-x 1 cvmfs cvmfs  266 Sep  1  2017 jupyter-nbconvert
-rwxr-xr-x 1 cvmfs cvmfs  265 Sep  1  2017 jupyter-nbextension
-rwxr-xr-x 1 cvmfs cvmfs  264 Sep  1  2017 jupyter-notebook
-rwxr-xr-x 1 cvmfs cvmfs  285 Sep  1  2017 jupyter-run
-rwxr-xr-x 1 cvmfs cvmfs  269 Sep  1  2017 jupyter-serverextension
-rwxr-xr-x 1 cvmfs cvmfs  269 Sep  1  2017 jupyter-troubleshoot
-rwxr-xr-x 1 cvmfs cvmfs  297 Sep  1  2017 jupyter-trust
-rwxr-xr-x 1 cvmfs cvmfs  422 Sep  1  2017 mako-render
lrwxrwxrwx 1 cvmfs cvmfs   42 Jun 19  2018 pandoc -> /sw/rhel6-x64/util/pandoc-2.2.1/bin/pandoc
-rwxrwxr-x 1 cvmfs cvmfs  247 Sep  1  2017 pip3
-rwxrwxr-x 1 cvmfs cvmfs  247 Sep  1  2017 pip3.5
lrwxrwxrwx 1 cvmfs cvmfs    8 Sep  1  2017 pydoc3 -> pydoc3.5
-rwxrwxr-x 1 cvmfs cvmfs  117 Sep  1  2017 pydoc3.5
-rwxr-xr-x 1 cvmfs cvmfs  260 Sep  1  2017 pygmentize
lrwxrwxrwx 1 cvmfs cvmfs    7 Sep  1  2017 python -> python3
lrwxrwxrwx 1 cvmfs cvmfs    9 Sep  1  2017 python3 -> python3.5
-rwxrwxr-x 1 cvmfs cvmfs  15K Sep  1  2017 python3.5
lrwxrwxrwx 1 cvmfs cvmfs   17 Sep  1  2017 python3.5-config -> python3.5m-config
-rwxrwxr-x 1 cvmfs cvmfs  15K Sep  1  2017 python3.5m
-rwxrwxr-x 1 cvmfs cvmfs 3.1K Sep  1  2017 python3.5m-config
lrwxrwxrwx 1 cvmfs cvmfs   16 Sep  1  2017 python3-config -> python3.5-config
lrwxrwxrwx 1 cvmfs cvmfs   10 Sep  1  2017 pyvenv -> pyvenv-3.5
-rwxrwxr-x 1 cvmfs cvmfs  269 Sep  1  2017 pyvenv-3.5
\end{minted}

Compute environments are managed via dedicated Jupyter kernel startup scripts and standalone software environment modules that are also accessible independently of Jupyter via Mistral command line working sessions.

\begin{minted}[breaklines,breakanywhere]{shell-session}
$ ls /sw/rhel6-x64/jupyterhub/jupyter_kernels/kernels/
anaconda2_bleeding  anaconda3_bleeding  python_2.7.12  python_3.5.2
$ cat /sw/rhel6-x64/jupyterhub/jupyter_kernels/kernels/anaconda3_bleeding/kernel.json
{
 "display_name": "Python 3 bleeding edge (using the module anaconda3/bleeding_edge)",
 "language": "python",
 "argv": [
	"/sw/rhel6-x64/jupyterhub/jupyter_kernels/scripts/anaconda3_bleeding.sh",
  "{connection_file}"
 ]
}
$ cat /sw/rhel6-x64/jupyterhub/jupyter_kernels/scripts/anaconda3_bleeding.sh
!/bin/bash
source /etc/profile
module purge
module load cdo/1.9.0-gcc48
#jfe: anaconda3/bleeding_edge conflicts with this, dunno why. But
#netcdf 4.6.1 is available...
#module load netcdf_c/4.3.2-gcc48
module load texlive/2016
module load anaconda3/bleeding_edge
python -m ipykernel_launcher -f "$1"
\end{minted}

Apparently, the Python and Anaconda environments that are used as Jupyter kernels were updated several times after initial deployment of the JupyterHub, but active maintenance has ceased months to years ago.

\begin{minted}[breaklines,breakanywhere]{shell-session}
$ ls /sw/rhel6-x64/conda/ -lha
total 32G
drwxr-xr-x   8 cvmfs cvmfs 4.0K Mar 17 10:39 .
drwxr-xr-x 108 cvmfs cvmfs 4.0K Nov 13  2019 ..
drwx------  20 cvmfs cvmfs 4.0K Oct 19  2016 anaconda2-4.2.0-python-2.7.12_do_not_use_anymore
drwx------  20 cvmfs cvmfs 4.0K Apr 26  2017 anaconda2-4.3.1-python-2.7.13_do_not_use_anymore
drwxrwxr-x  23 cvmfs cvmfs 4.0K Jun 20  2018 anaconda2-bleeding_edge
-rw-r--r--   1 cvmfs cvmfs 3.2G Oct 26  2017 anaconda2-bleeding_edge.tar-2017-10-26
-rw-r--r--   1 cvmfs cvmfs 3.2G Jun 13  2018 anaconda2-bleeding_edge.tar-2018-06-13
-rw-r--r--   1 cvmfs cvmfs 3.2G Jun 15  2018 anaconda2-bleeding_edge.tar-2018-06-15
drwx------  20 cvmfs cvmfs 4.0K Oct 19  2016 anaconda3-4.2.0-python-3.5.2_do_not_use_anymore
drwx------  20 cvmfs cvmfs 4.0K Oct  8  2018 anaconda3-4.3.1-python-3.6.0_do_not_use_anymore
drwxrwxr-x  26 cvmfs cvmfs 4.0K Jan 29 15:02 anaconda3-bleeding_edge
-rw-r--r--   1 cvmfs cvmfs 3.3G Jun 13  2018 anaconda3-bleeding_edge.tar-2018-06-13
-rw-r--r--   1 cvmfs cvmfs 9.5G Jun 22  2018 anaconda3-bleeding_edge.tar-2018-06-22
-rw-r--r--   1 cvmfs cvmfs 9.5G Feb 20  2019 anaconda3-bleeding_edge.tar-2019-02-20
\end{minted}

\begin{minted}[breaklines,breakanywhere]{shell-session}
$ ls /sw/rhel6-x64/python -la
total 7.9G
drwxrwxr-x   9 cvmfs cvmfs 4.0K Nov 23  2018 .
drwxr-xr-x 108 cvmfs cvmfs 4.0K Nov 13  2019 ..
drwxr-xr-x   7 cvmfs cvmfs 4.0K Mar  6  2018 python-2.7.12-gcc49
drwxrwxr-x   6 cvmfs cvmfs 4.0K Oct 26  2016 python-2.7.12-gcc49.new2016-12-22
-rw-r--r--   1 cvmfs cvmfs 725M Jan 20  2017 python-2.7.12-gcc49.tar.2017-01-20
-rw-r--r--   1 cvmfs cvmfs 778M Feb 14  2017 python-2.7.12-gcc49.tar.2017-02-14
-rw-r--r--   1 cvmfs cvmfs 992M Feb 15  2017 python-2.7.12-gcc49.tar.2017-02-15
-rw-r--r--   1 cvmfs cvmfs 1.1G Jun 20  2017 python-2.7.12-gcc49.tar.2017-06-20
-rw-r--r--   1 cvmfs cvmfs 1.1G Sep 25  2017 python-2.7.12-gcc49.tar.2017-09-25
drwxrwxr-x   6 cvmfs cvmfs 4.0K Apr 21  2015 python-2.7.9-gcc49
drwxrwxr-x   7 cvmfs cvmfs 4.0K Feb  9  2016 python-2.7-ve0-gcc49
-rw-r--r--   1 cvmfs cvmfs 1.1G May 18  2017 python-2.7-ve0-gcc49.tar.2017-05-18
-rw-r--r--   1 cvmfs cvmfs 1.1G Jul 11  2017 python-2.7-ve0-gcc49.tar.2017-07-11
drwxrwxr-x   7 cvmfs cvmfs 4.0K Oct 27  2016 python-3.5.2-gcc49
-rw-r--r--   1 cvmfs cvmfs 690M May 19  2017 python-3.5.2-gcc49.tar
-rw-r--r--   1 cvmfs cvmfs  98M Nov 11  2016 python-3.5.2-gcc49.tar.2016-11-11
-rw-r--r--   1 cvmfs cvmfs 490M Nov 18  2016 python-3.5.2-gcc49.tar.2016-11-18
drwxrwxr-x   3 cvmfs cvmfs 4.0K Jun  9  2015 swift-2.4.0-gcc49
drwxr-xr-x   2 cvmfs cvmfs 4.0K Dec 22  2016 tmp
\end{minted}

\subsection{Jupyter control script}

(Provide details on what cases were investigated with the remote Jupyter manager solution.)


%==============================================================================

\bibliography{references.bib}

\end{document}
