\begin{abstract}
Several high-performance computing centers have started to provide Jupyter-based access solutions to their compute and storage infrastructure.
The Jupyter software itself covers a wide range of usage and deployment scenarios, and Jupyter interfaces are designed to allow for setting up highly customized user working environments.
While this is a key strength of the Jupyter software from a user perspective, it poses a challenge for Jupyter service providers and as a result Jupyter service implementations might not turn out to support user productivity in an optimal way.
Here, scientific user productivity requirements for Jupyter on high-performance computing systems are formulated and existing Jupyter-based solutions on (national) supercomputers are evaluated.
Suggestions are given that are targeted not only at maximizing the productivity of scientific target users, but also at minimizing system administrator workload.
This white paper provides a viewpoint of expert users with a background in numerical ocean modeling, big data analysis, and community-driven open source software development, who also deployed, maintained and provided support for Jupyter-based workflows in teaching and scientific research locally at their institute.
\end{abstract}
