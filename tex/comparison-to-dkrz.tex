\section{Jupyter at DKRZ}
\label{sect:jupyter-at-dkrz}

In the following, the Jupyter-based solutions provided by DKRZ\footnote{Deutsches Klimarechenzentrum} are investigated.
First, a brief overview on the available compute infrastructure is given, then the officially supported Jupyter-based access options are described.
The goal is to enable a comprehensive discussion of the JupyterHub@JSC from a more general Jupyter@HPC viewpoint.

\subsection{HPC system infrastructure}

DKRZ operates \href{https://www.dkrz.de/up/systems/mistral}{Mistral} (HLRE-3) which is a tier-2 HPC system \cite{Wissenschaftsrat2015, GaussAllianz2020} with Earth system researchers as target user group.
The system was originally installed in July 2015\footnote{https://www.dkrz.de/up/systems/mistral/configuration} and is planned to operate until mid-2021\footnote{https://www.dkrz.de/kommunikation/aktuelles/dkrz-verfuenffacht-supercomputing-leistung-mit-neuem-bullsequana-von-atos}.
The system currently consists of roughly 1600 Intel Xeon E5-2680v3 phase-1 and 1800 Intel Xeon E5-2695V4 phase-2 compute nodes.
A very small fraction, i.e. 21 compute nodes (about 0.6\% of the total system, available via the \verb|gpu| partition) additionally operates a pair of several different types of Nvidia Tesla GPUs\footnote{what types of GPUs?}.
The main part of the system, i.e. about 96.2\% of the HPC system resources are reserved for classic multi- and full-node batch compute tasks (\verb|compute| and \verb|compute2| partition), with only a very small fraction of 1.1\% being reserved for shared-node compute tasks (\verb|shared| partition), respectively.
Another about 1.3\% of the system is explicitely reserved for data processing tasks (\verb|prepost| partition), whereas the remaining about 0.9\% are reserved for XXX tasks (\verb|miklip| partition)\footnote{are these dedicated for a special compute project?}.
The system is supplemented by seven login nodes which are dedicated for file editing, source code compilation, and the preparation and monitoring of batch tasks\footnote{https://www.dkrz.de/up/systems/mistral/login-and-environment} only.
Interactive command line analysis tasks are supposed to happen on the five \verb|mistralpp| interactive nodes\footnote{add URL}, that can also directly be accessed via external SSH sessions.

\subsection{JupyterHub service}

\subsection{Jupyter control script}
