\section{Recommendations for Jupyter@JSC}

\subsection{HPC resources}

\begin{itemize}
  \item \emph{Provide information on queue waiting times:} We recommend Jupyter@JSC provides a way of estimating possible waiting times, e.g., by generating a graphical representation of the current system occupation.
  This could greatly enhance the way users can plan their interactive work.
\end{itemize}


\subsection{Documentation}

\begin{itemize}
  \item \emph{Consolidate documentation:} Make available the system-specific documentation in a way that does not need a login to the system, and provide an obvious way for users to contributed to the JSC specific docs.
  \item \emph{Leverage community documentation:} Refer to existing community documentation whenever / wherever possible rather than repeating or copying non-synchronized versions of central materials.
  \item completely removes or clearly marks as obsolete any outdated JSC specific docs documentation,
  \item \emph{Document evolution of Jupyter@JSC:} Provides a changelog on Jupyter@JSC implementation to guide existing users who need a quick overview of new possibilities or restrictions and remove remove or mark as obsolete any outdated JSC specific documentation.
  \item \emph{Document intended use:} Provide documentation about usage patterns / anti-patterns (e.g., when to use / not use login nodes / batch nodes).
  \item \emph{Document resource limitations:} Make sure users understand where / when CPU hours are billed, and clearly highlight possible resource limitations (e.g., walltime or memory limits for Jupyter spawned on login nodes).
  \item \emph{Highlight system-specific optimizations:} Provide documentation on where to find, and how and why to use system-optimized performance-critical libraries and modules that is able to inform users who bring or build their own environments.
\end{itemize}


\subsection{Working environments}

\subsubsection{Pre-installed software}

\begin{itemize}
  \item \emph{Make optimized libraries easy to re-use:} For system specific low-level software that is critical for performance, we recommend the Jupyter@JSC system provides easy to use builds in a way that enables users in bringing / porting their own software selection based on these. E.g., if there is system-optimized builds of modules like numpy or cuda, make it easy to use them in own environments without forcing users to use other parts of the default envs.
  \item \emph{Remove or restrict the Lmod plugin:} We recommend, the Lmod plugin is removed or to keep restricted to only provide the Jupyter-specific part of the module tree.
  \item \emph{Actively support pip and conda:} To accomodate the users who come from the Python data analysis world, there should be an easy way of using the community standards (mainly pip and conda) for defining working environments.
\end{itemize}

\subsubsection{Configurability of working environments}

As was argued above, the separation between the Jupyter server and the Jupyter kernel is far from perfect.
This can result in conflicts between kernel and Jupyter server if, e.g., for reproducibility reasons, users keep un-changed kernels for a long time while the default JupyterLab setup is evolving, or if users use the full configurability of the kernels and create conflicts with the current default JupyterLab.

\begin{itemize}
  \item \emph{Make Jupyter user servers fully configurable:} To overcome this conflict, we recommend that Jupyter@JSC provides full configurability of not only the Jupyter kernels but also of the Jupyter user server.
  \item \emph{Allow for spawning old configurations of the defaults:} To also accomodate users with reproducibility demands but who do not intend to completely manage their own environments, we recommend Jupyter@JSC makes available frozen or checkpointed versions of the default Jupyter environments to be spawned via the hub.
\end{itemize}

\subsubsection{Conrtainer-based Jupyter and kernels}

\begin{itemize}
  \item \emph{Allow for container-based kernels:} Containers provide a way of isolating and porting working environments.
  Hence, they can be used for defining kernels (see \ref{sect:container-based-environment}) that are fully isolated from pre-defined environments.
  \item \emph{Allow for fully container-based Jupyter user servers:} There is ways of building user-defined full working environment including the Jupyter user server\footnote{CITE BINDER / LINK TO repo2docker} that are supported by the Jupyter community.
  These could solve virtually all of the above recommendations.
\end{itemize}
