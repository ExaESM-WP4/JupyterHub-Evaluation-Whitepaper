
\subsection{Working environments}

Here we summarize the status quo of working environment selection, modification and specification for JUWELS JupyterLab instances and give specific suggestions to achieve better robustness and flexibility, compatible with the chosen system software architecture.

A detailed description of the technical details of the JUWELS Jupyter setup is given in the appendix~\ref{app:tech-details-jsc-hub} with suggestions for simpler implementations of user-defined kernels (see~\ref{sect:custom-kernels}) based on Conda and containers are given.

Principal other user working environment managing approaches are discussed at the end of the report. OR SHOULD WE NOT DO THIS? (JupyterLab instances could also be started inside containers. Currently, this approach is used only on the HDF cloud. Should this be made possible on JURECA and JUWELS, too? What is pro and contra here? From a user and system administrator perspective?)

\subsubsection{Documented functionality for selection, modification and specification of working environments}

At JSC, working environment selection is provided via Jupyter kernels.
Per default, tehre are currently a Bash, Javascript, Julia and Python kernel, and several C++ kernels available.
The default Python package environment can be extended and modified using the Lmod software environment Jupyter extension available in the JupyterLab sidebar.

There is a range of problems with the module based approach of letting the user modify their environment.
First, there is no documentation of typical usage patterns and anti-patterns and no information on what the user can or cannot expect to achieve via the software environment module extension.
Then, most of the available Python modules are not compatible with the default environment, and users are esily getting to a state of broken Python kernels or even a defunct Jupyter instance (see~\ref{sect:lmod-use-and-problems}).
Finally, the documented ways of creating custom Jupyter kernels with a virtual Python environment hardly goes beyond slight modifications of an extensive default Python environment.

\subsubsection{Undocumented ways of specifying working environments}

Specifying a more standalone and light-weight virtual Python environment as Jupyter kernel is currently not documented but can easily be done (see~\ref{sect:suggested-python-virtual-environment}).
The specification of completely standalone working environments has been explored in sections \ref{sect:conda-environment} and \ref{sect:container-based-environment}.
Both a Conda package manager and a container based Jupyter kernel approach are suggested.
They extend the documented approaches by the possibility of deploying isolated software environments, enabling convenient porting of analysis environments and facilitating scientific reproducibility.

\subsubsection{Further customization}

\begin{itemize}
  \item Use own notebook server?
  \item Possible to use jupyter-server-proxy?
  \item Jupyter Plugins / Widgets?
\end{itemize}

\subsubsection{Suggestions}

\begin{itemize}

  \item Set up a software modules hierarchy that is scoped only for the JUWELS JupyterLab functionality and ensures the provided Lmod software environment modules are completely exchangeable.
  The user should not have a way of or should be aware of how to avoid  destroying their Jupyter kernels and JupyterLab instances.

  \item On system start-up, the \verb|Jupyter/2019a-Python-3.6.8| module automatically establishes the software environment for all available Jupyter kernels.
  To prevent version conflicts, it might make sense to provide only a very minimalistic Jupyter environment and prevent the user from unloading the modules that provide basic system functionality.
  Loading environment modules that provide Jupyter kernel functionality should be documented in a way the lets users consciously manage their software environments.
  This could, however, need modifications of the Jupyter kernel launcher tab, because currently, the launcher tab only shows the Jupyter kernels available on JupyterLab instance start-up and kernels that become available after JupyterLab has started can only be chosen from the kernel drop down of already open Jupyter notebooks.

  \item Another approach to create robust Jupyter kernels would be to make module-based kernel launcher scripts the default, let these show up in the JupyterLab launcher tab on system start-up.
  Loading and unloading of software environment modules would not have an effect on the kernel software environments and customization of kernel software environments would be done entirely by setting up own kernel launcher scripts.
  In this case good documentation would need to be provided by the system designers/administators and the Lmod software modules extension in the JupyterLab sidebar would become obsolete.

  \item Provide only a minimalistic Python interpreter software environment module for users to build their own virtual Python environments and Jupyter kernels.
  This would combine the advantage of using a JUWELS specific Python interpreter (with its potentially higher code execution speeds, see \href{https://www.fz-juelich.de/SharedDocs/Downloads/IAS/JSC/EN/slides/supercomputer-ressources-2019-11/15a-tuning_intel.html}{this example investigation by Intel}) with robust standalone pip package dependency managing.

  \item Extend the documentation for working environment specification based on the alternatives presented in section \ref{sect:jupyter-kernel-suggestions}.

  \item It should be noted that a complete and clear separation of JupyterLab and kernel functionality is not possible with the Jupyter ecosystem.
  Hence, a one-size-fits-all JupyterLab configuration that seamlessly works with every conceivable kernel cannot be provided.
  A possible way out would be to allow users to not only modify or create kernels, but to allow for full configuration also of upyterLab.
  This could be done with containers (which then would not only contain a Jupyter kernel but the full JupyterLab and kernel application) or by letting users specify virtual or Conda environments that also provide the Jupyter instance that is spawned by the JupyterHub.

\end{itemize}


